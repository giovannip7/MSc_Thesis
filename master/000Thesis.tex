% A LaTeX template for MSc Thesis submissions to 
% Politecnico di Milano (PoliMi) - School of Industrial and Information Engineering
%
% S. Bonetti, A. Gruttadauria, G. Mescolini, A. Zingaro
% e-mail: template-tesi-ingind@polimi.it
%
% Last Revision: October 2021
%
% Copyright 2021 Politecnico di Milano, Italy. NC-BY

\documentclass{Configuration_Files/PoliMi3i_thesis}

%------------------------------------------------------------------------------
%	REQUIRED PACKAGES AND  CONFIGURATIONS
%------------------------------------------------------------------------------

% CONFIGURATIONS
\usepackage{parskip} % For paragraph layout
\usepackage{setspace} % For using single or double spacing
\usepackage{emptypage} % To insert empty pages
\usepackage{multicol} % To write in multiple columns (executive summary)
\usepackage{booktabs}
\usepackage{rotating}
\usepackage{multirow}
\usepackage{outlines}
\usepackage{lscape}
\usepackage{minted}
\usepackage{textgreek}
\setlength\columnsep{15pt} % Column separation in executive summary
\setlength\parindent{0pt} % Indentation
\raggedbottom  
\usepackage[pygopt={texcomments=true,style=emacs}]{pythontex}
\setpythontexlistingenv{listing}

\newcounter{sublisting}[listing]
\newcommand{\codeline}[1]{%
  \addcontentsline{lopytx}{listing}%
    {\protect\numberline{\hspace{0.5in}\thelisting.\arabic{FancyVerbLine}}\hspace{0.5in}#1}%
}

% PACKAGES FOR TITLES
\usepackage{titlesec}
% \titlespacing{\section}{left spacing}{before spacing}{after spacing}
\titlespacing{\section}{0pt}{3.3ex}{2ex}
\titlespacing{\subsection}{0pt}{3.3ex}{1.65ex}
\titlespacing{\subsubsection}{0pt}{3.3ex}{1ex}
\usepackage{color}

% PACKAGES FOR LANGUAGE AND FONT
\usepackage[english]{babel} % The document is in English  
\usepackage[utf8]{inputenc} % UTF8 encoding
\usepackage[T1]{fontenc} % Font encoding
\usepackage[11pt]{moresize} % Big fonts

% PACKAGES FOR IMAGES
\usepackage{graphicx}
\usepackage{transparent} % Enables transparent images
\usepackage{eso-pic} % For the background picture on the title page
\usepackage{subfig} % Numbered and caption subfigures using \subfloat.
\usepackage{tikz} % A package for high-quality hand-made figures.
\usetikzlibrary{}
\graphicspath{{./Images/}} % Directory of the images
\usepackage{caption} % Coloured captions
\usepackage{xcolor} % Coloured captions
\usepackage{amsthm,thmtools,xcolor} % Coloured "Theorem"
\usepackage{float}

% STANDARD MATH PACKAGES
\usepackage{amsmath}
\usepackage{amsthm}
\usepackage{amssymb}
\usepackage{amsfonts}
\usepackage{bm}
\usepackage[overload]{empheq} % For braced-style systems of equations.
\usepackage{fix-cm} % To override original LaTeX restrictions on sizes

% PACKAGES FOR TABLES
\usepackage{tabularx}
\usepackage{longtable} % Tables that can span several pages
\usepackage{colortbl}
\usepackage{hhline}
\usepackage{pgfplots}
% PACKAGES FOR ALGORITHMS (PSEUDO-CODE)
\usepackage{algorithm}
\usepackage{algorithmic}
\usepackage{listings}

% PACKAGES FOR REFERENCES & BIBLIOGRAPHY
\usepackage[colorlinks=true,linkcolor=black,anchorcolor=black,citecolor=black,filecolor=black,menucolor=black,runcolor=black,urlcolor=black]{hyperref} % Adds clickable links at references
\usepackage{cleveref}
\usepackage{hyperref}
\usepackage[square, numbers, sort&compress]{natbib} % Square brackets, citing references with numbers, citations sorted by appearance in the text and compressed
\bibliographystyle{abbrvnat} % You may use a different style adapted to your field

% OTHER PACKAGES
\usepackage{pdfpages} % To include a pdf file
\usepackage{afterpage}
\usepackage{lipsum} % DUMMY PACKAGE
\usepackage{fancyhdr} % For the headers
\fancyhf{}
\PassOptionsToPackage

% Input of configuration file. Do not change config.tex file unless you really know what you are doing. 
\input{Configuration_Files/config}

%----------------------------------------------------------------------------
%	NEW COMMANDS DEFINED
%----------------------------------------------------------------------------

% EXAMPLES OF NEW COMMANDS
\newcommand{\bea}{\begin{eqnarray}} % Shortcut for equation arrays
\newcommand{\eea}{\end{eqnarray}}
\newcommand{\e}[1]{\times 10^{#1}}  % Powers of 10 notation

%----------------------------------------------------------------------------
%	ADD YOUR PACKAGES (be careful of package interaction)
%----------------------------------------------------------------------------

%----------------------------------------------------------------------------
%	ADD YOUR DEFINITIONS AND COMMANDS (be careful of existing commands)
%----------------------------------------------------------------------------

%----------------------------------------------------------------------------
%	BEGIN OF YOUR DOCUMENT
%----------------------------------------------------------------------------

\begin{document}

\fancypagestyle{plain}{%
\fancyhf{} % Clear all header and footer fields
\fancyhead[RO,RE]{\thepage} %RO=right odd, RE=right even
\renewcommand{\headrulewidth}{0pt}
\renewcommand{\footrulewidth}{0pt}}

%----------------------------------------------------------------------------
%	TITLE PAGE
%----------------------------------------------------------------------------

\pagestyle{empty} % No page numbers
\frontmatter % Use roman page numbering style (i, ii, iii, iv...) for the preamble pages

\puttitle{
	title=Simulation and Testing of Navigation for an Autonomous Mobile Robot, % Title of the thesis
	name=Giovanni Porcellato, % Author Name and Surname
	course=Automation and Control Engineering - Ingegneria dell'Automazione, % Study Programme (in Italian)
	ID  = 10745779,  % Student ID number (numero di matricola)
	advisor= Prof. Matteo Matteucci, % Supervisor name
		coadvisor={Simone Mentasti},
	academicyear={2021-22},  % Academic Year
} % These info will be put into your Title page 

%----------------------------------------------------------------------------
%	PREAMBLE PAGES: ABSTRACT (inglese e italiano), EXECUTIVE SUMMARY
%----------------------------------------------------------------------------
\startpreamble
\setcounter{page}{1} % Set page counter to 1

% ABSTRACT IN ENGLISH
\chapter*{Abstract} 
This thesis presents an integrated module for the simulation of a robot in a virtual environment, the creation of a testing platform and an attempt to solve the problems encountered during testing. 
The proposed approach thus offers both a solution to the complications encountered when testing hardware and a tool for collecting data and creating statistics in an automated manner.

The creation of the simulator benefits the development team, since they can take advantage of a system in which they can comfortably test new algorithms, without the risk of damaging the robot or its environment.

The creation of the testing platform serves the purpose of having solid data on which future performance improvements can be analytically evaluated.

Finally, the development of the image filter, responds to the need to solve problems related to certain light conditions that prevented the navigation stack from functioning correctly and consequently caused a degradation in performance.

\textbf{Keywords:} robotics, simulation, navigation, computer vision, image filter, testing platform % Keywords

% ABSTRACT IN ITALIAN
\chapter*{Abstract in lingua italiana}
Questa tesi presenta un modulo integrato per la simulazione di un robot in ambiente virtuale, la creazione di una piattaforma testing e un tentativo di soluzione ai problemi riscontrati in fase di testing. 
L'approccio proposto offre quindi al contempo una soluzione alle complicanze che si incontrano quando si ha a che fa testing su hardware e uno strumento per raccogliere dati e creare statistiche in modo automatizzato.

La creazione del simulatore beneficia il team di sviluppo, dal momento che può sfruttare  un sistema in cui testare comodamente i nuovi algoritmi, senza rischio di danneggiare il robot o l'ambiente circostante.

La creazione della piattaforma di testing risponde allo scopo di avere dei dati solidi su cui poter valutare i futuri miglioramenti delle performance in modo analitico.

Infine, lo sviluppo del filtro d'immagine, risponde all'esigenza di risolvere delle problematiche relative a certe condizioni di luce che impedivano il corretto funzionamento dello stack di navigazione e di conseguenza che causavano un degradamento delle performance.

\textbf{Parole chiave:} robotica mobile, simulatore, navigazione, visione computazionale, test

%----------------------------------------------------------------------------
%	LIST OF CONTENTS/FIGURES/TABLES/SYMBOLS
%----------------------------------------------------------------------------

% TABLE OF CONTENTS
\thispagestyle{empty}
\tableofcontents % Table of contents 
\thispagestyle{empty}
\cleardoublepage

%-------------------------------------------------------------------------
%	THESIS MAIN TEXT
%-------------------------------------------------------------------------
% In the main text of your thesis you can write the chapters in two different ways:
%
%(1) As presented in this template you can write:
%    \chapter{Title of the chapter}
%    *body of the chapter*
%
%(2) You can write your chapter in a separated .tex file and then include it in the main file with the following command:
%    \chapter{Title of the chapter}
%    \input{chapter_file.tex}
%
% Especially for long thesis, we recommend you the second option.

\addtocontents{toc}{\vspace{2em}} % Add a gap in the Contents, for aesthetics
\mainmatter % Begin numeric (1,2,3...) page numbering

% --------------------------------------------------------------------------
% NUMBERED CHAPTERS % Regular chapters following
% --------------------------------------------------------------------------
\chapter{Introduction}
\label{intro}
\input{01Chapter1}

\chapter*{Theoretical background}
From now on some basic knowledge needed to comprehend the reasoning of this work's study object is described and explained. 
This section consists of:
\begin{itemize}
    \item \textbf{Chapter \ref{ch:ros}:} \nameref{ch:ros}
    \item \textbf{Chapter \ref{ch:robot}:} \nameref{ch:robot}
    \item \textbf{Chapter \ref{ch:nav_stack}:} \nameref{ch:nav_stack}
\end{itemize}

\chapter{ROS}
\label{ch:ros}
\section{Introduction}
Platforms are becoming more and more significant in robotics. The two sorts of platforms are software platforms and hardware platforms. Low-level device control, SLAM (Simultaneous Localization and Mapping), navigation, manipulation, object or person recognition, sensing and package management, debugging and development tools are just a few of the characteristics that make up robot software platforms. These features are mostly used in the industrial sector, where robot software platforms are currently used most frequently. Robot hardware platforms include both industrial products and research platforms including humanoids, drones, and mobile robots. As a result, robotics researchers from around the world are working together to develop an open-source, user-friendly platform. Robot Operating System, sometimes known as ROS, is the most widely used robot software platform. The Robot Operating System, or ROS, is an open source framework for controlling the actions, motions, and other aspects of robots. In addition to those who have just started using robots, ROS is designed to be a software platform for both those who regularly develop and use robots.
\begin{figure}[H]
    \centering
    \includegraphics[scale=0.35]{Images/Chapter 2/ros_consortium.jpg}
    \caption{ROS Industrial Consortium}
    \label{fig:ros_consortium}
\end{figure}
\newpage
\section{History of ROS}
In the 1970s, the first specialized programming languages for robots emerged.
Robot-specific data types and libraries of robot functions existed. They did not permit integrated simulation, multi-robot interaction, or hardware abstraction. Standardization and code reuse were nonexistent.
Through the 1980s, 1990s, and particularly in the 2000s, when there was a strong drive to standardize robot components, their interfaces, and their fundamental functions, efforts to develop robot programming systems persisted.
\\
ROS was therefore first created in 2007 at the Stanford Artificial Intelligence Laboratory, where it is still used today. It has been controlled by OSRF since 2013, and it is now utilized by numerous robots, academic institutions, and businesses, becoming the de facto norm for robot programming.

% \begin{figure}[H]
%     \centering
%     \includegraphics[scale=0.16]{Images/Chapter 2/PR2_Robot_Willow_Garage_6-1.jpeg}
%     \caption{Willow Garage PR2 robot}
%     \label{fig:PR2}
% \end{figure}
\section{Meta-Operating System}
ROS is merely a middleware because it is an open source, robot-specific operating system.
A middleware is a piece of software that acts as a layer between the operating system and the applications, providing the developer with an additional degree of abstraction.
It simply acts as a mediator between software parts, allowing for easier communication.
Its function is to give an abstraction model for functions and the low-level implementation at the same time.
Each middleware product must offer:
\begin{itemize}
    \item Portability: common programming model regardless the programming language and system architecture
    \item Complexity management: low-level aspects are managed by libraries and drivers inside the middleware itslef
    \item Reliability: middleware allows robot developer to discard low level details
    \item Abstraction from sensors/actuators hardware;
    \item Communication protocol for data transport
\end{itemize}
 As a result, they are crucial to the creation of sophisticated programs that rely on a variety of hardware and software resources. \\
They still need to go through a lot of development before they can offer a full range of capabilities for general-purpose robots.
 
\begin{figure}[H]
    \centering
    \includegraphics[scale=0.5]{Images/Chapter 2/middleware.png}
    \caption{Meta Operating System}
    \label{fig:metaoperating}
\end{figure}
 In recent years, a number of robotic middlewares (OROCOS, ORCA, YARP, BRICS, etc.) were put forth; eventually, ROS emerged. 

 \subsection{Phylosophy of ROS}
Some philosophical facets of ROS are described in the following sentences:\\
\begin{itemize}
\item \textit{Peer to peer}: ROS systems are composed of a small number of interconnected computer applications that are continuously exchanging messages. These communications flow directly between programs. The system becomes more complex as a result, but as the amount of data grows, the system balances better.\\
\item \textit{Distributed}: Programs can be run on multiple computers and comunicate over the network.
\item \textit{Multilingual}: ROS decided on a multilingual strategy. Any language that has a client library created for it can be used to create ROS software modules. Client libraries are available for C++, Python, LISP, Java, JavaScript, MATLAB, and other programming languages as of this writing. \\
\item \textit{Thin}: contributors are encouraged to build standalone libraries by the ROS conventions. and then wrap those libraries, so that they can send and receive messages to and from other ROS modules. This additional layer is proposed to allow the reuse of software developed outside of ROS for other applications, and it greatly simplifies the development of automated tests using standard continuous integration tools..\\
\item \textit{Free and open source}: Free and open source: The permissive BSD license is used to release the ROS core, which allows both commercial and non-commercial use. ROS foresees data exchange between modules using inter-process communication (IPC), which means that systems built using ROS can have fine-grained licensing of their various components.\\
\end{itemize}

\section{ROS Architecture}
Based on a graph architecture, ROS allows for processing to occur in nodes, which exchange messages with one another synchronously by calling services, much like RPCs, and asynchronously by using topics to which they can subscribe and/or publish. In terms of structure, ROS is created on three conceptual levels:
\begin{enumerate}
    \item File-system level
    \item Computational level
    \item Community level
\end{enumerate}
We'll look at each level's individual components and how they fit into the architecture.
\begin{figure}[H]
    \centering
    \includegraphics[scale=0.35]{Images/Chapter 2/Filesystem.png}
    \caption{File System level representation}
    \label{fig:Filesystem}
\end{figure}
\subsection{File-system Level}
The Filesystem Level includes all resources used in ROS, in particular
\begin{itemize}
    \item Packages
    \item Metapackages
    \item Manifest
    \item Message types
    \item Service types
\end{itemize}

\textbf{Packages}\\
Packages are the main structure for organising ROS, \citet{rospackages}. These files include all of the data needed by the system during runtime, including processes, libraries, configuration files, datasets, and other files. They make up the structural elements of a ROS-based system. The package is represented by a directory at the filesystem level. 

There are some subfolders within the framework to manage the elements in order to encourage its growth:
\begin{itemize}
    \item \textit{include/packagename}: C++ include headers (make sure to export in the CMakeLists.txt)
    \item \textit{msg}:Folder containing Message (msg) types
    \item \textit{src/packagename}: Source files, especially Python source that are exported to other packages.
    \item \textit{srv/}: Folder containing Service (srv) types
    \item \textit{scripts}: executable scripts
    \item \textit{CMakeLists.txt}: CMake build file
    \item \textit{package.xml}:XML file containing package structure 
    \item \textit{CHANGELOG.rst}: Many packages will define a changelog which can be automatically injected into binary packaging and into the wiki page for the package
\end{itemize}
\textbf{Metapackages}
Metapackages are specialised structures whose only task is to represent a group of packages that have common characteristics with each other. The metapackages that are created in the context of older versions of ROS and later updated may also result from the conversion of older stacks that perform similar functions in the context of older versions of ROS, \citet{rosmetapackages}.\\
\newline
\textbf{Manifest}\\
A package manifest consists of an XML file named package.xml that must be included in the root folder of any catkin-compliant package. It contains information about the package, including its name, version number, authors, maintainers, and dependencies on other catkin packages. There is a strong similarity between this concept and the manifest.xml file used in the legacy rosbuild build system. System package dependencies are declared in package.xml, \citet{rosmanifest}.\\
There are a minimal set of tags that need to be nested within the <package> tag to make the package manifest complete.
\begin{itemize}
    \item \textit{<name>}: The name of the package
    \item \textit{<version>}: The version number of the package;
    \item \textit{<description>}: A description of the package contents;
    \item \textit{<maintainer>}: The name of the person(s) that is/are maintaining the package;
    \item \textit{<license}: The software license under which the code is released.
\end{itemize}
\textbf{Message types}\\
Message types define the structure of messages sent by ROS, \citet{rosmsg}. A separate sort of message is represented by each file with the.msg extension. Each line in the file corresponds to a message field. Each line has two columns: one for the field's data type (Int32/int (C++/Phyton), bool, string, time, etc.), and the other for the field name. The fields contained within these le can have values assigned to them; in this case, we're talking about constants. Msg file example in C++:

\textbf{Service types}\\
Service type are files that define the structure of request/response for ROS services, \citet{rossrv}.
These directly build on the msg format to allow node-to-node communication. They are kept in specialized.srv files in a package's srv/ subdirectory.
Srv file example in C++:
\begin{minted}{cpp}
 bool add(beginner_tutorials::AddTwoInts::Request  &req,
             beginner_tutorials::AddTwoInts::Response &res)
    {
      res.sum = req.a + req.b;
      ROS_INFO("request: x=%ld, y=%ld", (long int)req.a, (long int)req.b);
      ROS_INFO("sending back response: [%ld]", (long int)res.sum);
     return true;
   }
\end{minted}

\subsection{ROS Computational Graph Level}
The peer-to-peer network of ROS processes that are working together to process data is known as the Computation Graph. Nodes, Master, Parameter Server, messages, services, topics, and bags are the fundamental Computation Graph concepts of ROS. Each of these components provides data to the Graph in a different manner, \citet{rosconcepts}.\\
\begin{figure}[H]
    \centering
    \includegraphics[scale=0.8]{Images/Chapter 2/computationgraph.png}
    \caption{Computation Graph}
    \label{fig:computationgraph}
\end{figure}
\textbf{Nodes}\\
A node is a process that performs computation. Nodes are combined together into a graph and communicate with one another using streaming topics, RPC services, and the Parameter Server, \citet{rosnodes}. In accordance with the concept of modularity of the system, each node will be connected to just one specific functionality. In order to make the system more reusable, manageable, and understandable, ROS actually discourages the development of "omnipotent" nodes that can do several tasks. \\
The utilization of nodes in ROS has several advantages for the entire system. Since crashes only affect certain nodes, there is more fault tolerance. Compared to monolithic systems, code complexity is decreased. \\

\textbf{Topics}\\
 The buses that enable message exchange between nodes are known as topics and have formal and distinctive names. They implement a method for publishing and subscribing, where nodes that are configured to transmit or receive messages can act as publishers or subscribers. Anonymity policies between the nodes clearly distinguish between data producers and users. A limit amount of messages for each topic may be maintained in the queue in case they accumulate; any extra messages are not added to the queue and are lost, \citet{rostopics}.\\
\newline
\textbf{Services}\\
A two-way communication tool between nodes is services. It is a method that expands on messages by giving users the option to continue listening to a particular node while also issuing commands to it in order to receive a structured response. Each service is initially documented in an.srv file, which also lists the arguments and the type of return data in addition to the name of the service.
The service is represented within the server node by a function that accepts two pointers to objects of the server class as inputs: one will contain the function parameters (Request), and the other will gather the return value, \citet{rosservice}\\
\newline
\textbf{Messages}\\
The nodes of the graph exchange messages in order to communicate. These could be straightforward primitive types (such as integer, string, char, etc.) or arrays, or they could be even more complex, with structures resembling those used in C.\\
\newline
\textbf{Bags}\\
The method through which ROS stores logs and keeps track of all communications exchanged within a subject is represented by bags. Once a topic has been assigned to the rosbag tool, every message that is exchanged is saved in a corresponding file with the.bag extension. It is highly helpful for storing sensor data because it enables the developer to make a sort of "black box" for the robot. Additionally, ROS has a playback tool that enables graphical interface playback and visualization of the acquired data.\\

\textbf{Master}\\
In ROS, the master is in charge of several tasks, including adding new nodes to the network, managing the connections among the nodes in the graph, routing messages, and allowing a node to access the services of another.
It is the brain of the program and can only be used by one master concurrently. If the file is implemented properly, it can be started using the roscore command or launched automatically when a node starts.
\begin{figure}[H]
    \centering
    \includegraphics{Images/Chapter 2/ROS-master-node-topic.png}
    \caption{Visualization of Master-Node-Topic relationship}
    \label{fig:master-node-topic}
\end{figure}

\textbf{Parameter server}\\
The parameter server is essentially a component of the master, allowing certain network API configurations to be shared publicly with all nodes. Although not particularly fast, it is still useful during the software testing phase. The parameters are named according to the standard ROS naming convention. This means that ROS parameters follow the same hierarchical structure as topics and nodes, \citet{rosparmserv}.

\subsubsection{Coordinate Frames and Transforms}
A robot typically has numerous 3D reference systems that change over time. All of these coordinate systems are kept in a tree structure by the ROS tf package. This concept is also required to understand how URDF files handle the various parent and child links later on. As a result, the tf package keeps track of all existing relationships between point co-ordinate frames and calculates transforms between them. Developers can also use the command view frames to view the transform tree for debugging purposes.

\begin{figure}[H]
    \centering
    \includegraphics[scale = 0.5]{Images/Chapter 2/robottf.png}
    \caption{Transform Frames of a robot}
    \label{fig:robottf}
\end{figure}

\begin{figure}[H]
    \centering
    \includegraphics[scale = 0.5]{Images/Chapter 2/robottftree.png}
    \caption{Transform Frames Tree of a robot}
    \label{fig:robottf}
\end{figure}
\section{ROS Tools and Simulators}
ROS includes built-in tools that can be used when developing robotic applications or delving deeper into certain issues.
Although other tools exist, the ones described below were the ones that were used the most during the Oversonic project: RViz (3D visualization tool), Rqt (framework that enables GUI tools: Rqt graph, which displays correlation between nodes and messages in graph form, and Rqt plot), and, finally, Gazebo, a 3D simulator that has long been used in the course of this project, will be highlighted.
\subsection{RViz: 3D Visualization Tool}
RViz is a 3D visualization tool built into ROS.
Its primary function is to visualize ROS messages and topics in three dimensions, assisting the user in displaying data and understanding how our system behaves.
When starting a new RViz window from scratch, a black 3D scene appears. The user can indeed customize the environment by changing global options (for example, the fixed frame that provides a static reference view) and grid settings.
It is possible to choose which features to display by ticking them directly from the left panel (picture \ref{fig:rvizgui}).
RViz can visualize topic from camera sensor, showing the images on a dedicated box. This feature applies to any kind of sensor that communicates via ROS to our robot, e.g. Lidar, tracking camera, RGB camera etc. 
\begin{figure}[H]
    \centering
    \includegraphics[scale = 0.5]{Images/Chapter 2/rviz.jpg}
    \caption{RViz GUI}
    \label{fig:rvizgui}
\end{figure}
\subsection{Rqt: ROS GUI Development Tool}
Rqt is a ROS software framework that implements various GUI tools as plugins.
Rqt graph, in particular, is extremely useful.
The primary goal of this tool is to visualize ROS nodes, topics, and messages to aid in debugging and understanding of the system. In fact, when using ROS, it is beneficial to display the current graph in order to better understand how the various nodes communicate and how messages are exchanged.
Rqt graph thus results useful in:
\begin{itemize}
    \item Having a global overview of the system
    \item Debugging code in case there exist issues in nodes communication (e.g. two nodes are not connected in reality or too many nodes publish on the same topic)
\end{itemize}
\begin{figure}[H]
    \centering
    \includegraphics[scale=0.3]{Images/Chapter 2/rqt_graph_turtlesim.png}
    \caption{Graph from turtlesim}
    \label{fig:rqtgraph}
\end{figure}
In figure \ref{fig:rqtgraph} two nodes have been initialized and from the graph we can see the path of the messages.
 \subsection{Gazebo Simulator}
Dealing with real robots means using physics labs, charging batteries, calibrating sensors and many other small tasks involving hardware.
In real-life cases, even the best robots break down periodically due to human error, wear and tear or structural defects.  This is where simulators come in: in simulation, we can faithfully simulate the actions that the robot will perform, without the disadvantages listed above.
It is also possible to model actuators and sensors either as ideal devices or by adding varying degrees of distortion, error or failure. Thus, the solution of simulating robots in a virtual environment is efficient and cost effective.
The problem of SLAM (simultaneous localisation and mapping) has always been one of the most important research topics in the community.
In response to this need, 'Stages', for example, with high computational capabilities and different kinematic configurations, built-in sensors have been developed over the years.
In the context of this paper, it was decided to use Gazebo, a 3D simulator that provides various models of robots, sensors, environments, offering faithful and reliable simulations thanks to its physics engine. Gazebo is in fact one of the best-known simulators used in open source robotics.
 \begin{figure}[H]
     \centering
     \includegraphics[scale=0.25]{Images/Chapter 2/gazebogui.png}
     \caption{Gazebo GUI}
     \label{fig:gazebogui}
 \end{figure}
To use the simulator, you can either download a pre-fabricated model from the Gazebo robots (such as TurtleBot, PR2, Pioneer2, and other well-known robots) or build your own robot model (we'll discuss SDF and URDF later).
In terms of robots, your model can be equipped with a variety of sensors, including a stereo camera, RGB camera, tracking, and contact sensors. The noise model can also be added to the sensors.
ROS is tightly integrated with Gazebo thanks to the gazebo ros package. This is a simulator plugin module that enables bi-directional communication between Gazebo and ROS.
Simulated sensors and physical simulation data are thus bidirectionally transmitted between the two platforms.
 \begin{figure}[H]
     \centering
     \includegraphics[scale=0.40]{Images/Chapter 2/GazeboSimArchitecture.png}
     \caption{Gazebo Sim Architecture}
     \label{fig:gazebosimarch}
 \end{figure}
 
\subsubsection{Robot Modelling Formats}
As mentioned above, one of the possible ways of describing a robot model is through the URDF.
The unified robot description format is a package containing a file in XML format.
A URDF file is written in such a way that each link of the robot is a child of some other parent link, with joints connecting each link. In turn, the joints are defined by an offset from the reference frame of the parent link and their axis of rotation. 
This creates a complete kinematics model. 
Below is reported a simple sample of the creation of a base link:
\begin{minted}{xml}
    <link name="base_link">
    <visual>
      <geometry>
        <cylinder length="0.6" radius="0.2"/>
      </geometry>
      <material name="blue"/>
    </visual>
    <collision>
      <geometry>
        <cylinder length="0.6" radius="0.2"/>
      </geometry>
    </collision>
    </link>
\end{minted}

When we are developing complex robot models, it can happen that the notation within the XML file becomes verbose and prone to confusion.
It is precisely for this reason that the XACRO (Macros XML) model was created, which is nothing more than an XML  that allows the design phase of the model to be divided into sub-parts, which are then imported one by one into the main file. 
This results in cleaner code reading and facilitates debugging.

Below is an example of importing arguments from an external file:
\begin{minted}{xml}
    <xacro:property name=”robotname” value=”R001” />
    <link name=”${robotname}s_leg” />
\end{minted}

To use the model created within the Gazebo environment (which, as previously stated, works with SDFormat files), first convert from XACRO to URDF and then from URDF to SDF. In fact, files in this format include a description of the world in which the robot will be placed, a number of simulated physical world features (static and dynamic objects, lighting, terrain, and even physics), and plug-in additions.
An SDF file defining a model from scratch is shown below.:

\begin{minted}{xml}
<?xml version='1.0'?>
<sdf version='1.9'>
  <model name='my_model'>
    ...
  </model>
</sdf>
\end{minted}







\chapter{Robot}
\label{ch:robot}
\section{Introduction}
\\
\\
\\
\begin{center}
    \textit{<<In the twenty-first century the \\
    robot will take the place which \\
    slave labor occupied in ancient \\ 
    civilization. There is no reason at \\ 
    all why most of this should not \\
    come to pass in less than a century, \\
    freeing mankind to pursue its \\
    higher aspirations.>>} \\ 
            \text{Nikola Tesla (1856 - 1943) }
\end{center}


\begin{center}
    \textit{<<Robots of the world! \\
    The power of
man has fallen!\\ A new world has
arisen:\\ the Rule of the Robots!
March!>>}\\
    \text{Rossum's Universal Robot (1920)}
\end{center}

Man has always spent his life working. Dangerous and degrading work has been the cause of death for many people for centuries. 
In this sense, there has always been a tendency to try to relieve man of the heaviest jobs by looking for machines or automatic systems to replace him.
In a sense, with the advent of the industrial revolution, we witnessed the first real process of robotizing in history.
On the other hand, with the evolution of discoveries in the medical field, the desire and curiosity arose in man to try to clone himself, artificially constructing his own like.
It is here that these two needs and tendencies come together in what we now call humanoid robots.
Indeed, humanoid robots are designed and built to replace humans in the most physical and repetitive tasks, in order to ensure greater well-being.

\newpage

\section{History of Robotics}
In recent years, the general public has become increasingly interested in robots and robotics research. New developments, e.g. robotic competitions, which "push beyond the boundaries of current technological
systems" (such as Defense Advanced Research Projects Agency (DARPA) in the
United States), especially in the area of robotics, have promised and delivered
fully integrated systems, \citet{robocomp}.\\
But the idea of creating intelligent, useful machines for humans has existed since the beginning of mankind.
In fact, ever since civilisation, one of the most unattainable desires and ambitions for mankind has been to create artifacts of his own image.\\
From a historical perspective, the first example that can be interpreted as such dates back to 3500 B.C., with the legend of the giant Talus, the slave forged by Hephaestus.
Continuing in time and reaching the Babylonians in 1400 B.C., we can observe the creation of the first automatic machine, the clepsydra water clock.
Continuing through the centuries, creations became more and more technologically advanced and jumping back to the 1500s, we encounter Leonardo da Vinci and his many inventions.
\begin{figure}[H]
    \centering
    \includegraphics{Chapter 3/davinchiknight.jpeg}
    \caption{Leonardo da Vinci’s mechanical knight: sketches on the right, rebuilt
and showing its inner workings on the left.}
    \label{fig:my_label}
\end{figure}
The concept of the robot then gradually entered people's minds thanks to this long process, but it was only in the 20th century that it took on a real physical connotation.
\newline
The term 'robot' was introduced in 1920 by the play 'Rossum's Universal Robot', by Karel Čapek: it derives etymologically from the Slavic root word 'robota' meaning subordinate labor.
\begin{figure}[H]
    \centering
    \includegraphics{Images/Chapter 3/rossumplay.jpg}
    \caption{A scene from Rossum's Universal Robot play, showing three robots}
    \label{fig:rossum}
\end{figure}
Later, during the middle of the century, the first research into the connection between human and machine intelligence was undertaken, marking the beginning of Artificial Intelligence (AI).
Between 1950 and 1980, Isaac Asimov wrote the so called "Three Laws of Robotics" in his book 'Runaround'. They are encoded in the "positronic brains" and are defined as follows, \citet{asimov}:
\begin{itemize}
\item A robot may not injure a human being or, through inaction, allow a human
being to come to harm.
\item A robot must obey the orders given to it by human beings, except where
such orders would conflict with the First Law.
\item A robot must protect its own existence as long as such protection does not
conflict with the First or Second Law.
\end{itemize}
Around those years, the first robots were created, they stemmed from the confluence of advances in two fields: numerically controlled machines for precision manufacturing and remote control to handle highly radioactive materials.
In fact, these two fields already featured modern applications of technologies such as mechanics, control, computational science and electronics.
The first robots were therefore master-slave arms, designed to reproduce the mechanics of the human arm but with rudimentary control and little perception.\\
During the second half of the century, the development of integrated circuits, digital computers and miniaturised components allowed terminal-controlled robots to be designed and developed.\\
In fact, in the 1980s, robotics was defined as the science that studies the connection between action and perception.In fact, action involves a locomotion apparatus that moves in the environment and a manipulation apparatus that performs actions, modifying its surroundings, thanks to special actuators and end-effectors.\\
Perception is then extracted from the sensors that provide information about the state of the robot (e.g. position and speed) and its surroundings (e.g. range and vision).
In the 1990s, research was further accelerated by the need to rely on robots to replace human presence in critical environments.\\
As we enter the new millennium, robots have undergone profound transformations both in their scope of use and in their shapes and sizes. 
\subsubsection{Humanoid Robots}
As reported in the article "Humanoid Robots: Historical Perspective, Overview and Scope", \citet{Siciliano2020}:\\
\\
"\textit{The long saga of humanoid robots in science fiction has influenced generations
of researchers, as well as the general public, and serves as evidence that people
are drawn to the idea of humanoid robots. Humans generally like to observe and
interact with one another. In their social behavior, people are highly attuned to
human characteristics, such as the sound of human voices and the appearance of
human faces and body motion. \\
Infants show preferences for these types of stimuli at
a young age, and adults appear to use specialized mental resources when interpreting these stimuli. By mimicking human characteristics, humanoid robots can engage
these same preferences and mental resources.\\
Throughout history, the human body and mind have inspired artists, engineers,
and scientists, using media as diverse as cave paintings, sculpture, mechanical toys,
photographs, and computer animation. \\
Humanoid robots serve as a powerful new
medium that enables the creation of artifacts that operate within the real world
and exhibit both human form and behavior. 
\\The field of humanoid robotics focuses
on the creation of robots that are directly inspired by human capabilities and/or
selectively imitate aspects of human form and behavior. Humanoids come in a
variety of shapes and sizes, from complete human-size legged robots to isolated
robotic heads with human-like sensing and expression.}"\\
Thus, humanoid robots were developed to be employed as multi purpose mechanical workers, and were designed to work alongside humans in daily tasks, being a support, living in the same environment and using the same tools.
It must also be considered that when the robot moves around in the work environment, there can be multiple risks for the worker; in this respect, a subfield of robotics, called cognitive robotics, has taken hold.
Indeed, robots can take advantage of the traditional communication methods used among humans to become more aware of their surroundings.
An even more ambitious aim is to interpret human gestures through vision (eye gaze, body language). On the other hand, this could put a human in a difficult relationship with the robot, modelled by the phenomenon called 'uncanny valley', a concept introduced in the 1970s by Masahiro Mori, a professor at the Tokyo Institute of Technology.
Masahiro in fact argues that:\\
"\textit{I have noticed that, in climbing toward the goal of making robots appear human, our affinity for them increases until we come to a valley, which I call the uncanny valley.}"
\begin{figure}[H]
    \centering
    \includegraphics[scale=0.8]{Images/Chapter 3/Mori_Uncanny_Valley.png}
    \caption{Mori Uncanny Valley}
    \label{fig:mori_uncanny_valley}
\end{figure}
Mori better explains this concept with the example of the prosthetic hand:
\\
"\textit{One might say that the prosthetic hand has achieved a degree of resemblance to the human form, perhaps on a par with false teeth. However, when we realize the hand, which at first site looked real, is in fact artificial, we experience an eerie sensation. For example, we could be startled during a handshake by its limp boneless grip together with its texture and coldness. When this happens, we lose our sense of affinity, and the hand becomes uncanny.}"
\\
On the other hand, many scientists and researchers in the robotics community see humanoid robots as a possibility to better investigate human nature itself.
A part from the roles mentioned above, a humanoid robot could work as an avatar for telepresence, test ergonomics and serve for any other  roles that a human can do.
Even though in the past decades, humanoids have only been applied in research field, times seem to be mature to put these robots on field and let them cooperate with humans.
\section{Robee: Oversonic Robotics configuration}
In order to make physical sense of the results obtained within this project, it is important to define what technologies were used and what materials made up Robee's hardware.
\subsection{Hardware components and software architecture}
It is important to bear in mind that the Oversonic project has an architecture split between the
robot (also referred to as the edge) and the cloud, and these two components coexist in a
hybrid.
Describing the system from the cloud, the hardware component consists of a scalable node pool based on
the 2.35Ghz AMD EPYC 7452 processor that can achieve a boosted maximum frequency
of 3.35GHz with 32 GB RAM memory, running a Kubernetes instance on top of Linux
Ubuntu 18.04 (Bionic Beaver).
As far as the robot is concerned, all the computational power is provided by 2 Intel NUCs 8 including an Intel Core i5-8259U Processor (6M Cache, up to 3.80 GHz), 8 GB RAM and Integrated Graphics Intel Iris Plus 655.
The operating system which is mounted on is Linux Ubuntu 20.04 (Focal Fossa), and all the modules are running containers that on turn are managed by KubeEdge, a containers orchestration system built on Kubernetes.
\begin{figure}[H]
    \centering
    \includegraphics[scale=0.28]{Images/Chapter 3/oversonicarch.png}
    \caption{Oversonic Architecture}
    \label{fig:oversonicarch}
\end{figure}
\textbf{Internet of Things}\\
In the case of the Robee project, the architecture is therefore composed of various software modules that are containerised and must be able to communicate with each other.
The MQTT protocol is an optimal choice for this case.\\
From the official MQTT.org site: "\textit{MQTT is an OASIS standard messaging protocol for
the Internet of Things (IoT). It is designed as an extremely lightweight publish/subscribe
messaging transport that is ideal for connecting remote devices with a small code footprint and minimal network bandwidth. MQTT today is used in a wide variety of industries, such as automotive, manufacturing, telecommunications, oil and gas}", \citet{mqtt}.\\
MQTT therefore operates at the application layer of the OSI model, relying on TCP at the transport layer.
The MQTT protocol establishes two kinds of entities in the network: a message broker and a number of clients. The broker is nothing more than a server that receives all messages from all clients and then routes these messages to the relevant destination clients. A client is anything that can interact with the broker to exchange messages. The messages are routed to clients basing
on topics: every message is published over a specific topic, and only the clients subscribed
to it will receive the message.\\ A client, therefore, can be an IoT sensor or an application in a data centre that processes IoT data.
Each MQTT message has a command and a payload. The command defines the type of message:
\begin{itemize}
    \item CONNECT: initial message sent from client to broker, to instantiate a new connection
    \item DISCONNECT: final message sent from client to broker to end the connection
    \item PUBLISH: command to publish a message over a specified topic, it is sent from client to broker and then routed from broker to every client that appears to be subscribed to that topic
    \item SUBSCRIBE: message sent from client to broker in order to request a subscription to a specified topic
\end{itemize}
All MQTT libraries provide simple ways to handle such messages directly and can automatically populate certain required fields, such as 'message' and 'client Id'.
\begin{figure}[H]
    \centering
    \includegraphics[scale = 0.8]{Images/Chapter 3/mqtt.png}
    \caption{The MQTT publish and subscribe model for IoT sensors}
    \label{fig:mqtt}
\end{figure}
\subsection{Robots Configurations}
The robots covered by the work in this thesis are mainly three 
\begin{itemize}
    \item R007 is a small autonomous mobile robot used by Oversonic as a prototype in the testing phase and features skid steering kinematics. In fact, it has two belts with two torque motors. The system is based on an Intel NUC and peripherals: two or four lidar sensors, a tracking camera and a depth camera mounted on the top base. The use of this AMR (autonomous mobile robot) is mainly conceived in conjunction with its larger 'brother' Robee or in industrial logistics environments.
    \begin{figure}[H]
        \centering
        \includegraphics[scale=0.03]{Images/Chapter 3/r007.jpg}
        \caption{R007}
        \label{fig:r007}
    \end{figure}
    \newpage
    \item R012 is the humanoid robot developed in Oversonic Robotics, now in its fourth evolution from the initial prototype and featuring a differential drive base. The system is divided into two parts, a lower body and an upper body, each featuring an NUC terminal and several sensors, but in this analysis we will focus exclusively on the lower part. 
    The lower body in fact contains within it the two torque motors, which move two wheels actively. For the robot's stability, two passive caster wheels have been added to the front and rear of the base. Two Lidar sensors are then mounted on the base and going up about halfway up the torso are a tracking and depth camera. 
    \begin{figure}[H]
        \centering
        \includegraphics[scale=0.10]{Images/Chapter 3/r012.PNG}
        \caption{Robee R012}
        \label{fig:r012}
    \end{figure}
    \item N002 is a robot used for the testing phase of Robee's lower body. It has a similar purpose of use to R007.
    It has a differential drive base with two torque motor actuators, like R012 but with a physical arrangement of peripheral sensors but only one NUC.
    \begin{figure}[H]
        \centering
        \includegraphics[scale=0.10]{Images/Chapter 3/n002.jpg}
        \caption{N002}
        \label{fig:N002}
    \end{figure}
\end{itemize}
The main features of the skid steering and differential drive kinematics will be listed.
\textbf{Skid Steering}

Skid Steering is a particular kinematics configuration featuring two tracks. It is composed of two tracks on its basic configuration, left and right, and the control variables are indeed left and right speed.
Another configuration entails a 4 wheels set up featuring a low wheelbase configurations so that they can be deemed as two tracks.
When rotating the central point does not move as the track is sliding on the ground. It is clear that this drive needs proper calibration and slippage modeling in order to be reliable.
Some assumptions are needed for this kinematics model: the first, mass is placed in center of the fictitious medium, the second, all the wheels on the same side have the same speed.
While in motion, this kind of drive presents multiple ICR and all of them share the same $\omega_{z}$.
\begin{equation}
\begin{bmatrix}
    v_{x}\\
    v_{y}\\
    \omega_{z}
\end{bmatrix}
= J_{\omega} \begin{bmatrix}
\omega_{l}r\\
\omega_{r}r
\end{bmatrix}\end{equation}

 The wheels are turning and sliding simultaneously, resulting in two fictitious instantaneous centers of rotation: $ICR_{left}$ and $ICR_{right}$.
 Under proper assumptions, skid-steering can be simplified to a differential drive kinematics.
\begin{figure}[H]
    \centering
    \includegraphics[scale=0.40]{Images/Chapter 3/skidsteer2.png}
    \caption{Skid Steering Kinematics}
    \label{fig:skidsteer}
\end{figure}
 Under proper assumptions, skid-steering can be simplified to a differential drive kinematics.
 \newpage
\textbf{Differential Drive}

Differential drive configuration present the following construction:
\begin{itemize}
    \item two wheels working on the same axis
    \item two independent motors, one for each wheel
    \item one or two passive caster wheels
\end{itemize}

Control input in this case are the linear and the angular velocity of the robot, \textit{v} and {\textomega}.
Wheels move around an Istantaneous Centre of Rotation on a circular path with istantaneous radius R and angular velocity \textomega.

\begin{equation}
    \begin{bmatrix}
    x' \\
    y' \\
    \theta \\
    \end{bmatrix} = \begin{bmatrix}
    \cos{\omega \delta t} & -\sin{\omega \delta t} & 0 \\
    \sin{\omega \delta t} & \cos{\omega \delta t} & 0 \\
    0 & 0 & 1
    \end{bmatrix}\begin{bmatrix}
    x - ICR_{x} \\
    y - ICR_{y} \\ 
    \theta'    \end{bmatrix} + \begin{bmatrix}
    ICR_{x} \\
    ICR_{y} \\
    \omega \delta t
    \end{bmatrix}
\end{equation}

It is therefore possible to reconstruct robot pose from direct kinematics:

\begin{equation}
    x(t) = \frac{1}{2}(\int_{0}^{t} (V_{R}(t') + V_{L}(t')) \cos{\theta(t')} \,dt')
\end{equation}

\begin{equation}
    y(t) = \frac{1}{2}(\int_{0}^{t} (V_{R}(t') + V_{L}(t')) \sin{\theta(t')} \,dt')
\end{equation}

\begin{equation}
    \theta = \frac{1}{b}(\int_{0}^{t} (V_{R}(t') - V_{L}(t')) \,dt')
\end{equation}


where $V_{R}$ and $V_{L}$ are defined as follows:
\begin{equation}
    V_{R} = \omega (R + \frac{b}{2}) 
\end{equation}

\begin{equation}
        V_{L} = \omega (R - \frac{b}{2})
\end{equation}

and as a consequence the following is derived:
\begin{equation}
    V = \frac{V_{R} + V_{L}}{2}
\end{equation}
\begin{equation}
    \omega = \frac{V_{R} - V_{L}}{2}
\end{equation}

It becomes clear that we can compute robot odometry by integrating the so-called control variables and knowing the parameters of the wheels, namely the direct kinematics.
On the contrary, we can derive control variables from a desired pose or velocity.


\begin{figure}[H]
    \centering
    \includegraphics[scale=0.10]{Images/Chapter 3/diffdrive.png}
    \caption{Differential Drive Kinematics}
    \label{fig:diffdrive}
\end{figure}

It is interesting to mention three borderline cases for this kind of drive:
\begin{itemize}
    \item $V_{L}$ = $V_{R}$: forward linear motion is straight
    \item $V_{L}$ = - $V_{R}$: rotation in place
    \item $V_{L}$ = 0 or $V_{R}$ = 0: respectively, rotation about left and right wheel
\end{itemize}
\subsection{Sensors}
In order for a robot to perceive the world around it and to complete tasks autonomously, sensors are required. We distinguish between proprioceptive (internal state of the robot) and exteroceptive (state of the external environment) sensors. In this section, we focus on the exteroceptive sensors that have been used in Robee, providing a brief overview of their functioning.

\textbf{D400 Intel Depth Camera}

Depth cameras are a type of sensor widely used in robotic applications. They normally consist of two parts: a traditional digital camera, which captures RGB data, and a projector, which captures depth data.The depth system can work in several ways, for example by projecting a grid of light structured in a non-visible spectrum into a scene and then analysing the distortion created in this pattern to determine the distance and/or shape of any object placed in front, \citet{JONASSON2021112691}.
In our application, the main reason for using the d455 camera in Robee's lower body is ????distance measurement?????.
Traditional digital cameras shoot out an image as a grid of pixels in two dimensions. Each pixel is then associated with three values ranging from 0 to 255, which define the red, green and blue components, so black, for example, is (0,0,0) and a pure bright red would be (255,0,0). . This type of representation is called an RGB image. Thus, each image is composed of three channels each storing the values pertaining to each colour component.
\begin{figure}[H]
    \centering
    \includegraphics[scale=0.25]{Images/Chapter 3/rgb.jpeg}
    \caption{Channel decomposition}
    \label{fig:rgb}
\end{figure}
In the case of a depth camera, on the other hand, the pixels have different numerical values associated with them, where the number represents the distance of the corresponding pixel from the camera, thus the depth.
Thus, by unifying this representation we will have a colour map where cooler colours represent closer obstacles, and warmer colours represent more distant obstacles, in depth.

\begin{figure}[H]
    \centering
    \includegraphics[scale=0.6]{Images/Chapter 3/depthmap.png}
    \caption{Depth-map representation}
    \label{fig:depthmap}
\end{figure}

\begin{figure}[H]
    \centering
    \includegraphics[scale = 0.5]{Images/Chapter 3/inteld455.jpeg}
    \caption{D455 Intel Tracking Camera }
    \label{fig:d455}
\end{figure}

Therefore, there are two types of three-dimensional image formats, the first being RGB-D and the second being pointcloud.
The first has already been introduced above, and we recall that for each pixel, identified by the coordinates (x,y), four properties (R,G,B,depth) are associated.
The substantial difference between the point cloud and RGB-D data is that in the pointcloud, the coordinates (x,y) represent the real world value instead of integer values. When viewing the two types of data, in fact, the former is presented in a sparse structure, while the latter is based on grid-aligned images. 
A practical application of point cloud will be provided in chapter \ref{ch:pcl}
\begin{figure}[H]
    \centering
    \includegraphics[scale=0.4]{Images/Chapter 3/pointcloudsample.png}
    \caption{Point Cloud sample}
    \label{fig:pointcloudsample}
\end{figure}
Thus, the point cloud can be constructed from RGB-D format images. In fact, by knowing an RGB-D dataset and the camera's intrensic values through a process called camera calibration. 
Since pointclouds are sets of disordered vectors, it is common for researchers to change the structure of the pointcloud data into 3D voxel grids.
The voxel grid geometry is in fact a grid of values in three dimensions, organised in layers of rows and columns.
The reason for this conversion also comes from the fact that one often then has to deal with deep learning models that expect highly regular input data formats.
\begin{figure}[H]
    \centering
    \includegraphics[scale=0.25]{Images/Chapter 3/voxelgrid.png}
    \caption{Voxel Grid example: A 3D Convolutional Neural Network for Real-Time Object Recognition}
    \label{fig:voxel}
\end{figure}


\textbf{T265 Intel Tracking Camera}

The T265 tracking camera is designed to integrate odometry data from the robot. It is in fact an independent and robust support for visual-inertia odometry and re-localisation.
A key strength of visual-inertial odometry is that the various sensors available complement each other. The images from the visual sensors are supplemented by data from an onboard inertial measurement unit (IMU), which includes a gyroscope and accelerometer. The aggregated data from these sensors is fed into simultaneous localization and mapping (SLAM) algorithms.
The tracking is done by comparing the information collected by the two fish-eye cameras, which collect images at 30 fps, \citet{inteltracking}.
% \begin{figure}[H]
%     \centering
%     \begin{subfigure}
%             \includegraphics[scale=0.5]{Images/Chapter 3/t265diagram.jpg}
%     \caption{Block diagram of Intel T265}
%     \label{fig:t265block}
%     \end{subfigure}%
%     ~
%     \begin{subfigure}
%         \includegraphics[scale=0.5]{Images/Chapter 3/t265.jpg}
%         \caption{T265 Intel Tracking Camera used in Robee}
%         \label{fig:t265}
%     \end{subfigure}
% \end{figure}
\begin{figure}[H]
\centering
\begin{tabular}{cc}
\subfloat[Block diagram of Intel T265]{\includegraphics[scale=0.6]{Images/Chapter 3/t265diagram.jpg}}&
\subfloat[T265 Intel Tracking Camera used in Robee]{\includegraphics[scale=0.5]{Images/Chapter 3/t265.jpg}}\\

\end{tabular}
\end{figure}
\newpage
\textbf{YDLidar}

LiDAR (Light Detection And Ranging) identifies technology that measures the distance to an object by illuminating it with laser light, while at the same time being able to return high-resolution three-dimensional information about the surrounding environment. A LiDAR typically uses several components: lasers, photodetectors and readout integrated circuits (ROICs) with time-of-flight (TOF) capability to measure distance by illuminating a target and analysing the reflected light.

\begin{figure}[H]
\centering
\begin{tabular}{cc}
\subfloat[Lidar architecture]{\includegraphics[scale=0.7]{Images/Chapter 3/lidar.jpg}}&
\subfloat[Ydlidar model G4 used in Robee]{\includegraphics[width = 3in]{Images/Chapter 3/ydlidar.jpg}}\\
\end{tabular}
\end{figure}
YDLIDAR G4 is a 360-degree two-dimensional rangefinder 
developed by YDLIDAR. Based on the principle of triangulation, it is equipped with related optics, electricity, and algorithm design to achieve high-frequency and high-precision distance measurement. The mechanical structure rotates 360 degrees to continuously output the angle information as well as the point cloud data of the scanning environment while ranging. YDlidar provides a built in bridge to ROS that ease the integration with the overall system.
% Please add the following required packages to your document preamble:
% \usepackage{multirow}
\begin{table}[H]
\centering
\begin{tabular}{|c|c|c|c|c|}
\hline
\textbf{Item}                              & \textbf{Min} & \textbf{Typical} & \textbf{Max} & \textbf{Unit} \\ \hline
\textbf{Ranging Frequency}                 & /            & 9000             & /            & Hz            \\ \hline
\multirow{3}{*}{\textbf{Ranging distance}} & 0.12         & /                & 16           & m             \\ \cline{2-5} 
                                           & 0.26         & /                & 16           & m             \\ \cline{2-5} 
                                           & 0.28         & /                & 16           & m             \\ \hline
\textbf{Field of View}                     & /            & 0-360            & /            & Deg           \\ \hline
\textbf{Angle Resolution} &
  \begin{tabular}[c]{@{}c@{}}0.2 \\ at 5 Hz\end{tabular} &
  \begin{tabular}[c]{@{}c@{}}0.28 \\ at 7 Hz\end{tabular} &
  \begin{tabular}[c]{@{}c@{}}0.48 \\ at 12 Hz\end{tabular} &
  Deg \\ \hline
\end{tabular}
\caption{Specifications  for Ydlidar G4}
\label{tab:ydlidar}
\end{table}
\subsubsection{Visual Fiducial System}
During the use and development of Robee's navigation, so-called Apriltags were used, a particular visual fiducial system chosen for its robustness and integration with the simulated Gazebo environment.
Visual fiducials are nothing more than artificial landmarks, designed to be easily recognisable within the working environment and distinguishable from one another.
The methodology is similar to that of a common QR code but with significant applications and objectives.
In fact, unlike a QR code, where the user has to frame the tag with the camera and capture the high-resolution snapshot, these types of tags are designed to work with a small amount of information (even as little as 12 bits) but with performance and ease of use that is clearly superior to QR codes. In fact, these types of tags are designed to be automatically detectable and localisable even in low resolution conditions, even providing the relative position and orientation of the tag with respect to the camera.
In terms of size, the Apriltags used range from 50 to 100 pixels, including the payload, \citet{olson2011tags}.
\begin{figure}[H]
    \centering
    \includegraphics[scale=0.25]{Images/Chapter 3/apriltag.jpg}
    \caption{AprilTag distance and orientation measurement}
    \label{fig:apriltag}
\end{figure}

\begin{figure}[H]
    \centering
    \includegraphics[scale=0.6]{Images/Chapter 3/apriltagsteps.png}
    \caption{How AprilTag detection is performed}
    \label{fig:apriltagsteps}
\end{figure}
Visual fiducial systems have been used in robotics to improve human/machine interaction, enabling the development of commands such as 'follow me'.
In the context of this work, tags were used for SLAM (Simultaneous Localisation and Mapping), as an independent support to the sensors mentioned in the previous section.
In fact, in both the simulated and real environment, tags were placed in strategic positions in order to reposition the robot to the correct pose by comparing the information coming from the sensors and that coming from the tags.
AprilTag provides a package for perfect ROS integration.
\begin{figure}[H]
    \centering
    \includegraphics{Images/Chapter 3/aprilros.png}
    \caption{ROS TF}
    \label{fig:aprilros}
\end{figure}

The apriltag ROS package then takes as input the topic of the rectified image and returns a list of recognised tags and their positions in 3 dimensions.
However, in order to work, one must specify in the configuration files (settings and tags) which tag families to search for.
A practical application will be provided in chapter \ref{ch:sim_test}

\subsubsection{Indoor Positioning System}

For the purpose of this thesis, another technology that has been used is the Indoor Positioning System.
This is in fact a new approach to the problem of localisation when remote GNSS satellites, which are commonly blocked indoors, are not available.
There is now a wide offer for this type of solution, even with different communication protocols at its base: from Wi-Fi signals to Bluetooth to ultrasound.
In Robee, the choice was made to use an IPS system provided by Marvelmind, which is based on ultrasonic and time-of-flight (ToF) measurements with trilateration, \citet{yorke2021beacon}, and which also provides for communication via ROS topics, thus providing integration with the robotics platform used.
The system proposed by Marvelmind consists of a series of static beacons (four were used in our case), placed on the walls of the production area of Oversonic Robotics.

\begin{figure}[H]
    \centering
    \includegraphics[scale=0.25]{Images/Chapter 3/beacon.jpg}
    \caption{MarvelMind beacons kit}
    \label{fig:beacon}
\end{figure}

Each beacon sends and receives a stream of hypersonic signals continuously. There is also a modem, which is connected to the PC on which the supplied software is run, and a beacon called a 'hedgehog' which is placed on the vehicle to be located, in this case Robee. The hedgehog then receives the signals from the four beacons and sends them to the modem, which proceeds to triangulate.
The communication frequency is customisable and directly affects localisation accuracy, which in the basic configuration is claimed to be +/- 2 cm.


\chapter{Navigation Stack}
\label{ch:nav_stack}
\section{Introduction}
In this chapter, we will address SLAM (simultaneous localization and mapping) and navigation techniques used in this thesis. We will focus exclusively on the literature overview of techniques used in practice, without comparing the various existing approaches. We will therefore overview all the components of the see-plan-act architecture.
The sense-plan-act architecture in fact explains the entire process that starts from the map building, to the global and local planning up to sensor fusion.
It is indeed composed of:
\begin{itemize}
    \item Map
    \item Sensors 
    \item Current Position
    \item Goal Position
    \item Trajectory Planning
    \item Trajectory Following and Obstacle Avoidance
\end{itemize}

\begin{figure}[H]
    \centering
    \includegraphics[scale=0.21]{Images/Chapter 4/senseplanact.png}
    \caption{Block scheme representation of sense plan act architecture}
    \label{fig:my_label}
\end{figure}

\section{Simultaneous Localization and Mapping}
In this first subsection we will focus on map side, understanding how maps are build, interpreted and how the agents localize inside the map. There exists two main types of representing a map:
\begin{itemize}
    \item Landmark-based: a particular type of representation, mainly used for localization, that is based on detecting landmarks. This technique results in a sparse representation of the space, leaving much to the unknown;
    \item Grid maps: the map results in a discredited version of the environment, where each cell contain information about occupation/non occupation/unkwown. It results in a very dense representation where almost the totality of the cells are caught.
\end{itemize}

In the scope of this work we will focus on occupancy grid map.
\subsection{Occupancy Grid Map}
As anticipated, occupancy grid map is a peculiar map representation that attempts to discretize the continous environment into a two dimensional grid map. The grid map is again divided into array cells of size from 5 to 50 cm and each of them hold a probability value that stands for the likelihood to be free or occupied.
Thus, occupancy grid maps try to solve the problem of reconstructing consistent maps from noisy and uncertain measurement data, under the hypothesis of knowing the robot pose. 
The reasoning behind most occupancy grid mapping algorithm is to calculate the posterior over maps, given the data, in a probabilistic way.
\begin{equation}
    p(m | z_{1:t},x_{1:t})
\end{equation}
where m is the map,$z_{1:t}$ is the set of measurements up to time t and $x_{1:t}$ the set of all the poses taken by the robot, namely its path.
Being $m_{i}$ the i-th grid cell and having it a binary occupancy value that states if a cell is free or occupied (1 for the cell being occupied, 0 free), we can define:
\begin{equation}
m = \sum_{i}^{}{m_{i}}    
\end{equation}.
\begin{figure}[H]
    \centering
    \includegraphics[scale=0.5]{Images/Chapter 4/occupancygrid.png}
    \caption{Sample of an Occupancy Grid}
    \label{fig:occupancygrid}
\end{figure}
Due to the curse of dimensionality, the probabilistic approach reduces to estimate the single cell occupancy rather than the entire map:
\begin{equation}
    p(m_{i} | z_{1:t},x_{1:t})
\end{equation}

To calculate the single cell occupancy we resort to Bayes rule:
\begin{equation}
    p(m_{i} | z_{1:t},x_{1:t}) = \frac{p(z_{t} | m_{i},z_{1:t-1},x_{1:t}) p(m_{i} | z_{1:t-1},x_{1:t})}{p(z_{t} |z_{1:t-1},x_{1:t})}
\end{equation}

Additionally recurring to Markov assumption, stating that the current state depends on only a finite fixed number of previous states, measurement $z_{t}$ depends only on $x_{t}$ and $m_{i}$.
It is common use at this point to adopt log-odds representation of occupancy, as to avoid being with probabilities close to 0 and 1:
\begin{equation}
    l_{t,i} = \log{\frac{p(m_{i} | z_{1:t},x_{1:t})}{1 - p(m_{i} | z_{1:t},x_{1:t})}}
\end{equation}

The process loops and assumes the form of the algorithm \ref{alg:occupancy}.
It is important to note that only the cells which fall under the sensor cone of measurement are updated through the inverse sensor model, \citet{thrun2005probabilistic}.
\begin{algorithm}
\caption{Occupancy Grid Algorithm}\label{alg:occupancy}
\begin{algorithmic}
\STATE Algorithm occupancy grid mapping({$l_{t-1,i},x_{t},z_{t}$})
\FOR{all cells $m_{i}$}
    \IF{$m_{i}$ in perceptual field of $z_{t}$}
        \STATE $l_{t,i}$ = $l_{t-1,i}$ + $inverse\_sensor\_model(m_{i},x_{t},z_{t} - l_{0})$
    \ELSE
        \STATE $l_{t,i}$ = $l_{t-1,i}$
    \ENDIF
\ENDFOR
\RETURN  {$l_{t,i}$}
\end{algorithmic}
\end{algorithm}

where the inverse sensor model is defined as follows:
\begin{equation}
    inverse\_sensor\_model(m_{i},x_{t},z_{t}) = p(m_{i}|z_{t}, x_{t})
\end{equation}
The motivation for the "inverse" denomination is because it reasons from effects to causes: it provides an information about the world where that same information was derived from a measurement caused by the world it self:

\begin{equation}
    p(m_{i} | z_{1:t},x_{1:t}) = \eta \int{m:m(i)=m_{i}}^{} p(z | x,m) p(m) dm
\end{equation}

A function approximator has to be used, since this algorithm cannot be computed due to the large map space.

\subsection{SLAM algorithm}
At this point we turn to the problem of SLAM, Simultaneous Localisation and Mapping. This stems from the robot's need to map a new environment, of which nothing is known, and at the same time to localise the robot itself within the map being created. The problem is particularly difficult as one does not have access to the robot's poses and uncertainty is kept on all the components.
Moreover, it proposes to correct both odometry and uncertainty of estimated position and landmark.
Two approaches to SLAM can be defined, from a probabilistic point of view:
\begin{itemize}
    \item Full SLAM: simultaneous estimate of path and map
    \item Online SLAM: simultaneous estimate of the most recent pose and map
\end{itemize}

\textbf{Full SLAM}
Full SLAM addresses the problem of estimating the joint probability of the entire trajectory and landmark.
\begin{equation}
    p(x_{1:t}, m | z_{1:t},u_{1:t})
\end{equation}
For this purpose we propose a significant example: FastSLAM.
Fast Simultaneous Localization and Mapping uses a sampled particle filter distribution model, solving the full SLAM problem.
If we consider the full trajectory $X_{t}$ rather than a single pose $x_{t}$, the following holds:
\begin{equation}
    p(X_{t}, m | z_{t}) = P(X_{t}| z_{t}) P(m | X_{t}, z_{t})
\end{equation}
where $P(X_{t}| z_{t})$ is the estimate of the trajectory and $P(m | X_{t}, z_{t})$ is the estimate of the map given the trajectory.
Thus, in FastSLAM the trajectory $X_{t}$ is represented by particles $X_{t}(i)$ while the map is represented by a factorization called Rao-Blackwellized filter.
The approach so is to treat each pose particle as it it was the entire trajectory, processing all of the feature measurements independently.

\begin{equation}
    P(m | X_{t}^{(i)},z_{t}) = \prod_{j}^{M} P(m_{j} | X_{t}^{(i)},z_{t})
\end{equation}
Indeed, once the trajectory is known, all of the features become uncorrelated.

\begin{equation}
    p(x_{1:t},l_{1:m} | z_{1:t},u_{0:t-1}) = p(x_{1:t} | z_{1:t},u_{0:t-1})p(l_{1:m} | x_{1:t},z_{1:t}) 
\end{equation}
where we have SLAM posterior, robot path posterior and landmark positions respectively. Previous equation can be simplified by factorization as follows:

\begin{equation}
     p(x_{1:t},l_{1:m} | z_{1:t},u_{0:t-1}) = p(x_{1:t} | z_{1:t},u_{0:t-1})\prod_{i=1}^{M}p(l_{i} | x_{1:t},z_{1:t})
\end{equation}
In this way the dimension of state space is reduced making particle filtering possible:
\begin{equation}
    O(N\times\log(M))
\end{equation}
with N being the number particles and M the number of map features
\begin{figure}[H]
    \centering
    \includegraphics[scale=0.34]{Images/Chapter 4/slam_uncertainty.png}
    \caption{SLAM problem: initial uncertainty on pose and consequent decrease thanks to previously seen landmarks, \citet{thrun2004}}
    \label{fig:slam_uncertainty}
\end{figure}
Summarizing, FastSLAM adopts a Rao-Blackwellized particle filtering based on landmarks, \citet{montemerlo2002}, where each particle is a trajectory, each landmark is represented by a 2x2 EKF and therefore each particle has to maintain M EKFs.



\textbf{Online SLAM}
Online SLAM entails estimating the posterior over the last pose along with the map:
$p(x_{t},m_{i} | z_{1:t},u_{1:t})$ where $x_{t}$ is the pose at time t, m is the map, $z_{1:t}$ $u_{1:t}$ the measurements available up to time t.
The fact that we refer to this technique as online SLAM directly derives from the fact that we're considering data at time t, estimating last pose only.
\begin{equation}
    p(x_{t}, m | z_{1:t},u_{1:t}) = \int \int ... \int{}^{} p(x_{1:t}, m | z_{1:t}, u_{1:t}) dx_{1}dx_{2} ... dx_{t-1}
\end{equation}
As one can notice, the online SLAM problem is the result of integrating out one at a time past poses from the full SLAM problem.
A significant example of a proposed solution to Online SLAM is proposed: EKF SLAM.
Extended Kalman Filter Simultaneous Localization and Mapping uses a linearized Gaussian probability distribution.


\begin{algorithm}
\caption{Extended Kalman Filter}\label{alg:ekf_slam}
\begin{algorithmic}
\STATE Extended Kalman filter({$\mu_{t-1},\Sigma_{t-1},u_{t},z_{t}$})
\STATE {$\mu$}' = g($u_{t}$, $\mu_{t-1}$)
\STATE $\Sigma_{t}^{}$' = $G_{t}\Sigma_{t-1}^{} G_{t}^{T} + R_{t}$
\STATE $K_{t}$ = $\Sigma_{t}'H_{t}^{T}(H_{t}\Sigma_{t}'H_{t}^{T} + Q_{t})^-1$
\STATE $\mu_{t}$ = $\mu_{t}' + K_{t}(z_{t} - h{\mu_{t}})$
\STATE $\Sigma_{t}$ = $(I - K_{t}H_{t})\Sigma_{t}'$
\RETURN $\mu_{t}, \Sigma_{t}$
\end{algorithmic}
\end{algorithm}

EKF-SLAM promises good performances but it has two main drawbacks: it employs linearized models of non-linear motion and observation models, inheriting many caveats; it is computationally demanding.
One possible solution to this problem is the above reported FastSLAM (Rao-Blackwellisation filter).

\subsection{SLAM Toolbox}
During the experience at Oversonic Robotics, the SLAM toolbox was chosen for the simultaneous localisation and mapping problem, though many predecessors exist.
The ROS packages responsible of SLAM can be divided into Bayes-based filter implementations, like GMapping and HectorSLAM, and graph-based implementations, as Cartographer and Karto SLAM.
The SLAM Toolbox package is an open source software developed by Steve Maceski which use graph based approach and occupancy grid map.
It has been widely used on the various ROS distros and has become the default SLAM algorithm for ROS2. It arose from the need to build accurate maps of large environments, where previous SLAM tools had shown shortcomings.\\
SLAM Toolbox provides three operating modes, \citet{Macenski2021}:
\begin{itemize}
    \item Synchronous Mapping: provides the ability to map and localize in an environment while keeping a bunch of measurements to be added to SLAM. This results useful when the quality of the map is important.
    \item Asynchronous Mapping: on the contrary, this mode manages new measurements only when previous measurement has been completed. This makes this modality useful when real time localization is crucial.
    \item Pure Localization: cannot detects changes in the space. It tries to match a local bunch of measurements with the data originally gathered.
\end{itemize}

\subsection{Advanced Monte Carlo Localization}
The pose estimation used in Oversonic applications is based on AMCL pose (Advanced Monte Carlo Localization). 
This technique relies on the so called particle filter approach. An extremely large number of particles covering the entire state space are used to initialize particle filters. The robot has a multi-modal posterior distribution because it predicts and update the measurements as it obtains more measurements. A Kalman Filter approximates the posterior distribution to be a Gaussian, which is a significant departure from this. The particles converge to a single value in state space after several rounds. Maintaining the random distribution of particles over the state space is a major challenge for particle filters, which becomes impossible for large dimensional problems. These factors make an adaptive particle filter far superior to a simple particle filter in terms of convergence speed and computational efficiency. 
\begin{figure}[H]
    \centering
    \includegraphics[scale=0.5]{Images/Chapter 4/amclocalization.png}
    \caption{Particle Filter in Action over Progressive Time Steps, \citet{amclimage}}
    \label{fig:amclpart}
\end{figure}The main concept is to limit the inaccuracy caused by the particle filter's sample-based representation. The real posterior is thought to be represented by a discrete, piecewise constant distribution, such as a discrete density tree or a multidimensional histogram, in order to calculate this bound. We start with a map of our surroundings when using an adaptive particle filter for localization, and we can either manually localize the robot by setting it to a certain point or we may make the robot start from with no initial estimate of its position. We now create new samples that forecast the robot's position following the motion command as it advances. By re-weighting these samples and leveling the weights, sensor readings are included. In general, it is a good idea to add a few random, evenly dispersed samples because they aid in the robot's recovery when it loses track of its location. Without these random samples, the robot will continue to resample from the erroneous distribution under those circumstances and will never recover. We could encounter dis-ambiguities inside a map due to symmetry in the map, which is what gives us a multi-modal posterior belief, which is why it takes the filter many sensor readings to converge.

\section{Global Planning}
Mobile robots are meant to move from their current positions to some goal inside the map.
Once that the SLAM task has been performed and a map has been obtained, we can address the This is known as trajetory planning and it is managed by the so-called global planner.
Robot motion planning goals are:
\begin{itemize}
    \item collision-free trajectories
    \item most efficient or most optimal (depending on the chosen optimality criterion) trajectory
\end{itemize}
The problem that global planner addresses regards finding a collision free path between an initial pose and the goal, taking into account the existing constraints.
It is important to distinguish between some concepts used in this scope:
\begin{itemize}
    \item Path: a geometric locus of way points
    \item Trajectory: a path for which a temporal law is specified
    \item Manouver: a series of actions that a vehicle should execute
\end{itemize}

In the scope of this thesis we are going to analyze a path planning algorithms, the so-called A*.
This algorithms is part of the graph based planning family.
The underlying idea is to construct a discretized representation of the map, building a graph out of it (4 or 8 neighbors connectivity are possible) and eventually searching for the shortest path in the graph, namely the optimal solution. It is relevant to report that the resolution of the grid directly influences the accuracy of the plan: a more dense resolution will describe in a more complete way the map it is investigating, thus resulting in a deeper analysis of the possible paths.

\begin{figure}[H]
     \centering
     \subfloat[][a]{\includegraphics[scale=0.45]{Images/Chapter 4/4connectivity.png}\label{4 connectivity}}
     \hfill
     \subfloat[][b]{\includegraphics[scale=0.46]{Images/Chapter 4/8connectivity.png}\label{8 connectivity}}
     \caption{Comparison of 4 connectivity (a) and 8 connectivity (b)}
     \label{steady_state}
\end{figure}
This connectivity scheme reproduces on a grid the kinematics motion a robot is supposed to do in reality, so that performing the path search on the grid is representative of how the robot would move in reality. 
The usual approach to search graphs for the optimal path 
\subsection{A* algorithm}
The A* algorithm was developed on the basis of the Dijkstra algorithm m improving its performance and is therefore one of the most widely used in path finding and graph traversal today.
The components of this algorithm are the two points (start and end point), the grid and the nodes. 
In this approach what is important is the cost of moving from one edge to the others.
The predecessor of A*, Dijkstra algorithm, in fact focuses on the idea of cost: each part of the path has an intrinsic cost and the algorithm visits all of the existing edges trying to lower the overall cost. The algorithm manages a queue list where it keeps all the nodes that are still to analyzed, where the nodes with the smallest distance to the starting point is the first node in the queue, \citet{herzog}.
Every time we move to some node, we encounter other nodes that previously were unaccessible, since we are dealing with k connectivity framework, and if these nodes are still to be traversed they are added to queue. This process ends as soon as the priority queue is emptied, namely when there are no nodes left to be investigated.
Every time the algorithm shifts to some new node it records the path that led to that node and the specific costs, so the shortest path to each node can be computed by going backwards in the path.
A* is based on the best first search speeds up this process by splitting the cost into a function:
\begin{equation}
    f(x) = g(x) + h(x)
\end{equation}

where g(x) is cost of the shortest path from the starting point to the current node and h(x), the so-called heuristic function, is an estimate of the cost of the shortest path from the current state to the goal.
The kind of heuristic is a matter of choice, still it needs to comply with the following three properties:
\begin{itemize}
    \item Completeness: the algorithm is guaranteed to terminate when dealing with finite graphs having non negative edge weights.
    \item Admissibility: the heuristic never overestimates the cost of reaching the goal.
    \newpage
    \begin{equation}
        h(x) \le h^*(x)
    \end{equation}
    where h*(x) is defined as the optimal cost to reach a goal from the current node.
    
    \item Consistency: the estimate of the algorithm is always less than or equal to the estimated distance from any neighbouring node to the goal, plus the cost of reaching that node.
\end{itemize}
Time complexity strongly depends on the heuristic and in its worst case (the case in which the search space is unbounded), the number of nodes exploded is exponential in the depth of the solution:
\begin{equation}
    d: O(b^d)
\end{equation}
where b is defined as the branching factor.
In its best case (the search space is a tree and there exists only one goal) it would develop in a polynomial fashion, provided that the following condition on the heuristic holds:
\begin{equation}
    |h(x) - h^*(x)| = O(\log{h^*(x)})
\end{equation}
where h* is the optimal heuristic.
For this reason, a bounded relaxation is applied: it is possible to speed up the process by considering also approximate shortest paths. This process is bounded by a factor $\epsilon$ so that optimality  suffers a decrease that is not greater than (1 + $\epsilon$) times the optimal solution, computed without the hypothesis relaxation.
Several possible algorithms exists for $\epsilon$, below is reported as an example the Dynamic Weighting, \citet{10.5555/1624775.1624777}:\\
cost function is defined as
\begin{equation}
    f(n) = g(n) + (1 + \epsilon w(n))h(n)
\end{equation}
where w(n) is
\begin{equation}
    w(n) = 
    \begin{cases}
1 - \frac{d(n)}{N}, & \text{if}\ d(n) \le N \\
0 , & \text{otherwise}
\end{cases}
\end{equation}
with d(n) being the depth of the search and N the anticipated length of solution.
Below is reported the pseudocode of the $A^{*}$ algorithm:

\begin{algorithm}
\caption{A algorithm}\label{alg:a_star}
\begin{algorithmic}
\STATE Input: A graph G(V,E) with source node start and goal node end
\STATE Output Least cost path from start to end
\STATE open\_list = {start}
\STATE closed\_list = {}
\STATE g(start) = 0
\STATE h(start) = heuristic\_function(start,end)
\STATE f(start) = g(start) + h(start)
\WHILE{open list is not empty}
    \STATE m = Node on top of open\_list, with least f
    \IF{m == end}
        \RETURN
    \ENDIF
    \STATE remove m from open\_list
    \STATE add m to closed\_list
    \FOR{each n in child(m)}
    \IF{n in closed\_list}
    \STATE continue
    \ENDIF
    \STATE cost = g(m) + distance(m,n)
    \IF{n in open\_list and cost < g(n)}
    \STATE remove n from open list as new path is better
    \ENDIF
    \IF{n in closed\_list and cost < g(n)}
    \STATE remove n from closed list
    \ENDIF
    \IF{n not in open\_list and n not in\_closed list}
    \STATE add n to open\_list
    \STATE g(n) = cost
    \STATE h(n) = heuristic\_function(n, end)
    \STATE f(n) = g(n) + h(n)
    \ENDIF
    \ENDFOR
    \ENDWHILE
\RETURN failure
\end{algorithmic}
\end{algorithm}
It is interesting to note that the algorithm returns either the optimal plan, hence it is an exact algorithm.
\begin{figure}[H]
    \centering
    \includegraphics[scale=0.60]{Images/Chapter 4/Aproblem.png}
    \caption{A* initial problem: red point is the starting node, green point is the goal}
    \label{fig:aproblem}
\end{figure}

\begin{figure}[H]
     \centering
     \subfloat[][a]{\includegraphics[scale=0.60]{Images/Chapter 4/Asol1.png}\label{}}
     \hfill
     \subfloat[][b]{\includegraphics[scale=0.60]{Images/Chapter 4/Asol2.png}\label{}}
     \hfill
     \subfloat[][c]{\includegraphics[scale=0.60]{Images/Chapter 4/Asol3.png}\label{}}
     \caption{Shortest path relaxing the admissibility criteria}
     \label{a_sol}
\end{figure}
\begin{figure}[H]
     \centering
     \subfloat[][a]{\includegraphics[scale=0.60]{Images/Chapter 4/Arel_sol1.png}\label{}}
     \hfill
     \subfloat[][b]{\includegraphics[scale=0.60]{Images/Chapter 4/Arel_sol2.png}\label{}}
     \hfill
     \subfloat[][c]{\includegraphics[scale=0.60]{Images/Chapter 4/Arel_sol3.png}\label{}}
     \caption{Optimal path obtained using the admissibility criteria. }
     \label{a_sol}
\end{figure}
\section{Local Planning}
A strong assumption on which global planning technique was based is the fact that the space surrounding the robot from its starting point to its goal is almost totally known. Nonetheless, robots moving in every environment must be able to deal with unforeseen changes and adapt to them. For this purpose, a local planner is paired with the global planner: while the latter is responsible of trajectory planning and  works at a low frequency rate, the first one deals with trajectory following and obstacle avoidance at a much higher frequency rate.
So, local planning is deemed to solve two tasks:
\begin{itemize}
    \item Ensure path following
    \item Perform obstacle avoidance for objects that are not tracked in the map
\end{itemize}
\textbf{Obstacle Avoidance}
“Let A be the robot moving in the workspace W, whose configuration space is CS. Let
q be a configuration, $q_{t}$ this configuration in time t, $A(q_{t}) \in W$ the space occupied by the robot in this configuration.
If in the vehicle there is a sensor, which in qt measures a portion of the space $S(q_{t}) \subset W$ identifying a set of obstacles $O(q_{t}) \subset W$. Let u be a constant control vector and $u(q_{t})$
this control vector applied $q_{t}$ during time $\delta t$. Given $u(q_{t})$, the vehicle describes a trajectory \\
$q_t + \delta_t = f(u, q_t, \delta t)$, with        $\delta t \geq 0$.
Let $Q_t,T$ be the set of the configuration of the trajectory followed from $q_t$ with $\delta t \in (0,T)$ a given time interval. T > 0 is called the sampling period. Indicating with $q_target$ a target configuration. Then, in time $t_i$ the robot A is in $q_ti$, where a sensor measurement is obtained $S(q_{ti})$, and thus an obstacle description $O(q_ti)$.” \citet{SicilianoKhatib2008}
The overall goal of the obstacle avoidance algorithm is to find a trajectory from that brings the robot closer to the goal in a non colliding way:\\
$A(q_{ti}, T) \cap O(q_ti) = \emptyset$ \\
$f(q_{ti}, q_{target}) \le F(q_{ti} + T, q_{target})$\\
\begin{figure}[H]
    \centering
    \includegraphics[scale = 0.5]{Images/Chapter 4/obstacleavoidance.png}
    \caption{Sample of the most simple obstacle avoidance technique}
    \label{fig:obstacleavoidance}
\end{figure}
In the following subsections an overview of the most famous local planning method will be provided, in particular:
\begin{itemize}
    \item Vector Field Histogram
    \item Curvature Velocity
    \item Dynamic Window Approach
\end{itemize}
\subsection{Vector Field Histograms}
Vector Field Histogram was presented in 1991 by \citet{borenstein1991} and ensured fast obstacle detection and collision avoidance, not requiring the vehicle to stop. It is composed of two steps, where the first one all the possible motions are evaluate and in the second one the best one is traversed.
    \begin{figure}[H]
        \centering
        \includegraphics[scale=0.30]{Images/Chapter 4/vff1.png}
        % \hfill
        % \subfloat[]{}\includegraphics[scale=0.25]{Images/Chapter 4/vff2.png}
        \caption{Vector Field Histogram}
        \label{fig:my_label}
    \end{figure}
The Vector Field Histogram is based on the concept of virtual force field, a concept that resort somehow to the idea of imaginary forces acting on a robot, \citet{1087247}.
VFF is composed of:
\begin{itemize}
    \item A two-dimensional Cartesian histogram grid for obstacle representation, where the grid is composed of cells defined by some coordinate (i,j) and $c_{i,j}$ holds the probability of the occupancy. A probability distribution is created by updating only one cell in the histogram grid for each range reading.
    More specifically, the function $h^k(q_{ti})$ describes the density of the obstacle, on turn proportional to the probability of point occupancy P(p) and to distance from the obstacle, that is to say that the more the distance increases, the lower is the density.
    The function $h^k(q_{ti})$ is defined as:
    \begin{equation}
        h^k(q_{ti}) = \int_{\Omega_k} P(p)^n\left(1 - \frac{d(q_{ti},p)}{d_{max}}\right)^m dp
    \end{equation}
    \begin{figure}[H]
        \centering
        \includegraphics{Images/Chapter 4/potentialfield.png}
        \caption{Vector Field Histogram}
        \label{fig:my_label}
    \end{figure}
    \item Application of potential field idea to the histogram grid
    \item Previous two component are combined in real time enables sensor data to perform obstacle avoidance
\end{itemize}
All the directions ranged from the sensors are evaluated but only those that fall under the defined threshold are further investigated. A cost function is then established as follows:
\begin{equation}
    G = \alpha \times target\_direction + \beta \times  wheel\_orientation + \gamma \times previous\_direction
\end{equation}
where $\alpha$ represents the direction of the goal, $\beta$ the smoothing of the action and $\gamma$ the previous direction of motion.
Every direction, falling under the threshold, is evaluated through the newly defined cost function, that becomes the selection method.

\subsection{Curvature Velocity Methods}
Curvature Velocity Methods were developed by Reid Simmons in 1996.
This approach to obstacle avoidance treats the problem as a constrained optimization in the velocity space of the robot, rather than in Cartesian space.
The robot is deemed to travel along arcs of circles rather than straight lines, still it cannot turn instantaneously.
This method works by adding constraints to the velocity space, defined as the set of controllable velocities and choosing the point in that space that complies with all the constraints and maximizes an objective function, that balances speed, safety and goal-directedness, \citet{simmons}.
\begin{figure}[H]
    \centering
    \includegraphics[scale= 0.5]{Images/Chapter 4/vcm.png}
    \caption{Curvature Velocity Methods}
    \label{fig:cvm}
\end{figure}
Only the interval of curvatures between $c_{min}$ and $c_{max}$ are considered, where the set of considered trajectories is obtained by using the curvatures tangent to obstacles to divide the velocity space  into regions of constant distance.

\newpage
\subsection{Dynamic Window Approach}
The dynamic window approach is an obstacle avoidance technique developed by Dieter Fox, Wolfram Burgard and Sebastian Thrun in 1997. 
In the DWA approach, the search for commands to control the robot takes place directly in velocity space. Compared to the previously seen method, the robot's dynamics are also integrated, thus further constraining the velocity search space to those that respect the dynamics constraints and are safe with respect to the obstacle.
The process can be divided into search space and optimization, \citet{fox1997}.
The search space of the possible velocities is reduced in three points:
    \begin{itemize}
        \item Circulare Trajectories: only circular trajectories are considered, determined by pair of (v,w) translational and rotational velocities
        \item Admissible Velocities: restriction to consider only safe velocities
        \item Dynamic Window: further restriction that applies to the admissible velocities, selecting only those that can be reached within a short time interval, respecting the constraints on acceleration.
    \end{itemize}
The dynamic windows search space reduces to $V_r = V_s \cap V_a \cap V_d$\\

Optimization step proposes to maximize the objective function
\begin{equation}
    G(v,w) = \sigma(\alpha \times heading(v,w) + \beta \times dist(v,w) + \gamma \times vel(v,w))
\end{equation}
where \textit{heading} is a measure of progress towards goal location, \textit{dist} is the distance towards the closest obstacle and \textit{vel} is the the foward velocity of the robot.
\begin{figure}[H]
    \centering
    \includegraphics[scale=0.75]{Images/Chapter 4/dynamic_window.jpg}
    \caption{Dynamic Window}
    \label{fig:dwa}
\end{figure}

\begin{algorithm}
\caption{DWA algorithm}
\begin{algorithmic}
\STATE BEGIN DWA(robotpose, robotGoal, robotModel)
\STATE desired\_V = calculate\_V(robotPose,robotGoal)
\STATE   laserscan = readScanner()
\STATE   $allowable_v$ = generateWindow(robot\_V, robotModel)
\STATE   $allowable_w$  = generateWindow(robot\_W, robotModel)
\FOR{each v in $allowable_v$}
      \FOR{for each w in $allowable\_w$}
      \STATE dist = $find\_dist$(v,w,laserscan,robotModel)
      \STATE breakDist = calculateBreakingDistance(v)
      \IF{(dist > breakDist)}  
         \STATE heading = hDiff(robotPose,goalPose, v,w)
         \STATE clearance = (dist-breakDist)/(dmax - breakDist)
         \STATE cost = costFunction(heading,clearance, $abs(desired\_v - v)$)
         \IF{(cost > optimal)}
            \STATE $best\_v = v$
            \STATE $best\_w = w$
            \STATE optimal = cost
        \ENDIF
      \ENDIF
      \ENDFOR
\ENDFOR
\STATE  set robot trajectory to $best\_v, best\_w$
\END
\end{algorithmic}
\end{algorithm}
\newpage


\chapter*{Contribution}
\begin{itemize}
    \item \textbf{Chapter \ref{ch:sim_test}: \nameref{ch:sim_test}}
    \item \textbf{Chapter \ref{ch:pcl}: \nameref{ch:pcl}}
\end{itemize}

\chapter{Simulator and Testing Platform}
\label{ch:sim_test}
\section{Introduction}
\textbf{Motivation for the use of a simulator DA ADATTARE}
% In the last decade, one of the biggest drivers for success in machine learning has arguably been the rise of high-capacity models such as neural networks along with large datasets such as ImageNet to produce accurate models. While we have seen deep neural networks being applied to success in reinforcement learning (RL) in domains such as robotics, poker, board games, and team-based video games, a significant barrier to getting these methods working on real-world problems is the difficulty of large-scale online data collection. Not only is online data collection time-consuming and expensive, it can also be dangerous in safety-critical domains such as driving or healthcare. For example, it would be unreasonable to allow reinforcement learning agents to explore, make mistakes, and learn while controlling an autonomous vehicle or treating patients in a hospital. This makes learning from pre-collected experience enticing, and we are fortunate in that many of these domains, there already exist large datasets for applications such as self-driving cars, healthcare, or robotics.



\begin{equation}
        NP = W * X =
    \begin{bmatrix}
            w_{1} & w_{2} & w_{3} & w_{4}
    \end{bmatrix}
    \begin{bmatrix}
        & x_{1} &\\
        & x_{2} &\\
        & x_{3} &\\
        & x_{4} &\\
    \end{bmatrix}
\end{equation}
\begin{equation}
    W =
    \begin{bmatrix}
        15 & 30 & 30 & 25
    \end{bmatrix}
\end{equation}
\begin{equation}
    x_{1} = \frac{Navigated Distance}{Path Distance}
\end{equation}
\begin{equation}
    x_{2} = \frac{Forward Average Speed}{Requested Speed}
\end{equation}
\begin{equation}
    x_{3} = \frac{Target Navigation Time}{ Navigation Time}
\end{equation}
\begin{equation}
    x_{4} = \frac{Target Rotation Time}{Rotation Time}
\end{equation}
\begin{equation}
    NP = \frac{w_{1}*x_{1}+w_{2}*x_{2}+w_{3}*x_{3}+w_{4}*x_{4} }{w_{1} + w_{2} + w_{3} + w_{4}} 
    = \frac{15*x_{1}+30*x_{2}+25*x_{3}+30*x_{4}}{100}
\end{equation}

    \begin{gather}
        a_{21} = -\frac{k_{s}}{J_{1}} 
        \quad 
        a_{21} = -\frac{k_{s}}{J_{1}}
        a_{21} = -\frac{k_{s}}{J_{1}}
    \end{gather}

\chapter{Pointcloud Filter}
\label{ch:pcl}
\section{Introduction}

POINTCLOUD 
PCL LIBRARY
\newpage
\section{Problem Explanation}
\newpage
TILE FLOORS 
HIGH LIGHTS

\newpage
\section{Proposed Solution}
ALGO OR CODE 
RESULTS


\chapter{Conclusions and future developments}
\label{ch:conclusions}%
In the first part of this work, a simulator was developed from scratch to test new navigation algorithms. This made it possible to test the new custom recovery behaviour, for example, first in the simulated environment and once the goodness of the development was seen, it was implemented in the real robot. Consequently, a testing platform was developed that provides a comprehensive overview of performance measurements, motor current data and a robot trajectory visualiser. During testing, a problem was observed with the depth camera, which recognised reflections on the floor as obstacles. A real time filter was then implemented to decrease the density of the pointcloud data. Thanks to this, it was possible to solve the problem of phantom obstacles by decreasing the stuck time, in a computational time that does not affect the result. 
The simulator could be subject to further improvements in the future: one possible route that could be taken would be to break free from gazebo plug-ins, for example by modelling the differential drive on a par with that used in the real robot. Furthermore, in order to make the simulated model completely mirror the real one, another possibility that could be pursued would certainly be to analytically model friction and wheel slip. As far as the pointcloud filter is concerned, other possible solutions can be tested: there are several filters for manipulating the point cloud, the results of which have yet to be tested.

%-------------------------------------------------------------------------
%	BIBLIOGRAPHY
%-------------------------------------------------------------------------

\addtocontents{toc}{\vspace{2em}} % Add a gap in the Contents, for aesthetics
\bibliography{Thesis_bibliography} % The references information are stored in the file named "Thesis_bibliography.bib"

%-------------------------------------------------------------------------
%	APPENDICES
%-------------------------------------------------------------------------

\cleardoublepage
\addtocontents{toc}{\vspace{2em}} % Add a gap in the Contents, for aesthetics
\appendix
\chapter{Appendix A}
\label{ch:appA}
\section{XACRO code of the model}
\begin{minted}[fontsize={\fontsize{9}{10}\selectfont}]{xml}
<?xml version="1.0"?>
<robot xmlns:xacro="http://www.ros.org/wiki/xacro" name="navbot">

  <xacro:property name="base_width" value="0.3"/>
  <xacro:property name="base_length" value="0.5"/>
  <xacro:property name="base_height" value="0.15"/>
  <xacro:property name="wheel_radius" value="0.06"/>
  <xacro:property name="wheel_width" value="0.04"/>
  <xacro:property name="wheel_separation" value="0.38"/>
  <xacro:property name="wheel_joint_offset" value="0.02"/>
  <xacro:property name="caster_wheel_radius" value="0.03"/>
  <xacro:property name="caster_wheel_joint_offset" value="0.2"/>
  <xacro:property name="laser_radius" value="0.03"/>
  <xacro:property name="laser_len" value="0.04"/>

  <xacro:include filename="$(find nav-sim)/urdf/calculations.xacro" />
  <xacro:include filename="$(find nav-sim)/urdf/materials.xacro" />


<!-- BASE LINK -->
  <link name="base_link">
    <visual>
      <origin xyz="0 0 0" rpy="0 0 0"/>
        <geometry>
          <box size="0.001 0.001 0.001" />
        </geometry>
    </visual>
  </link>

  <link name="chassis">
    <xacro:box_inertia m="10" w="${base_width}" h="${base_height}" d="${base_length}"/>
    <collision>
      <origin xyz="0 0 ${base_height/2}" rpy="0 0 0"/>
      <geometry>
        <box size="${base_length} ${base_width} ${base_height}"/>
      </geometry>
    </collision>
    <visual>
      <origin xyz="0 0 ${base_height/2}" rpy="0 0 0"/>
      <geometry>
        <box size="${base_length} ${base_width} ${base_height}"/>
      </geometry>
      <material name="water"/>
    </visual>
  </link>

  <joint name="base_link_joint" type="fixed">
    <origin xyz="0 0 ${wheel_radius}" rpy="0 0 0" />
    <parent link="base_link"/>
    <child link="chassis"/>
  </joint>


<!--WHEELS -->

  <xacro:macro name="wheel" params="prefix reflect">
    <link name="${prefix}_wheel">
      <xacro:cylinder_inertia m="2" r="${wheel_radius}" h="0.005"/>
<!--      TODO: add rotation in inertia-->
      <visual>
        <origin xyz="0 0 0" rpy="${pi/2} 0 0"/>
        <geometry>
          <cylinder radius="${wheel_radius}" length="${wheel_width}"/>
        </geometry>
        <material name="orange"/>

      </visual>
      <collision>
        <origin xyz="0 0 0" rpy="${pi/2} 0 0"/>
        <geometry>
          <cylinder radius="${wheel_radius}" length="${wheel_width}"/>
        </geometry>
    <joint name="${prefix}_wheel_joint" type="continuous">
      <origin xyz="0 ${((wheel_separation/2))*reflect} 0"
      rpy="0 0 0"/>
      <parent link="chassis"/>
      <child link="${prefix}_wheel"/>
      <axis xyz="0 1 0"/>
    </joint>
  </xacro:macro>

  <xacro:wheel prefix="left" reflect="1"/>
  <xacro:wheel prefix="right" reflect="-1"/>


<!-- CASTER WHEELS -->

  <xacro:macro name="caster_wheel" params="prefix reflect">
    <link name="${prefix}_caster_wheel">
      <xacro:sphere_inertia m="1" r="${caster_wheel_radius}"/>
      <visual>
        <origin xyz="0 0 0" rpy="0 0 0"/>
        <geometry>
          <sphere radius="${caster_wheel_radius}"/>
        </geometry>
        <material name="black"/>
      </visual>
      <collision>
        <origin xyz="0 0 0" rpy="0 0 0"/>
        <geometry>
          <sphere radius="${caster_wheel_radius}"/>
        </geometry>
        <surface>
        <friction>
          <ode>
            <mu>0</mu>
            <mu2>0</mu2>
            <slip1>1.0</slip1>
            <slip2>1.0</slip2>
          </ode>
        </friction>
      </surface>
      </collision>
    </link>

    <joint name="${prefix}_caster_wheel_joint" type="continuous">
      <axis xyz="1 1 1"/>
      <parent link="chassis"/>
      <child link="${prefix}_caster_wheel"/>
      <origin xyz="${reflect*caster_wheel_joint_offset} 0 ${-wheel_radius+caster_wheel_radius}" 
      rpy="0 0 0"/>
    </joint>
  </xacro:macro>

  <xacro:caster_wheel prefix="front" reflect="1"/>
  <xacro:caster_wheel prefix="back" reflect="-1"/>


<!--LASERS-->
  <xacro:macro name="laser" params="suffix reflect">
    <link name="laser_frame_${suffix}">
      <visual>
        <origin xyz="0 0 0" rpy="0 0 0"/>
        <mass value="1" />
        <geometry>
          <cylinder radius="${laser_radius}" length="${laser_len}"/>
        </geometry>
      </visual>
      <collision>
        <origin xyz="0 0 0" rpy="0 0 0"/>
        <geometry>
          <cylinder radius="${laser_radius}" length="${laser_len}"/>
        </geometry>
      </collision>
      <xacro:cylinder_inertia m="1" r="${laser_radius}" h="${laser_len}"/>
    </link>
    <joint name="laser_joint_${suffix}" type="fixed">
      <origin xyz="${reflect*(base_length/2-laser_radius)} 0 ${base_height+laser_len/2}" rpy="0 0 0"/>
      <parent link="chassis" />
      <child link="laser_frame_${suffix}" />
    </joint>
  </xacro:macro>

  <xacro:laser suffix="HF" reflect="1"/>
  <xacro:laser suffix="HB" reflect="-1"/>


<!--  <xacro:include filename="$(find nav-sim)/urdf/_d435.urdf.xacro" />-->
<!--  <sensor_d435 parent="chassis">-->
<!--    <origin xyz="0 0 0" rpy="0 0 0"/>-->
<!--  </sensor_d435>-->

  <xacro:include filename="$(find nav-sim)/urdf/navbot_gazebo_plugins.gazebo.xacro"/>

</robot>

\end{minted}

\chapter{Appendix B}
\label{ch:appB}


\section{Testing Module}
\begin{minted}[fontsize={\fontsize{9}{10}\selectfont}]{python}
#!/usr/bin/env python
from math import sqrt
from pathlib import Path
import rospy
from geometry_msgs.msg import Twist
from nav_msgs.msg import Odometry
from openpyxl import Workbook
from openpyxl import load_workbook
from cros.topics import ROSTopics
rospy.init_node("measures")
rospy.loginfo("Started measuring ...")
RATE = 10  # Hz
rate = rospy.Rate(RATE)
# filename = "../measurement.xlsx"  # save in isc_slam
rounding = 2
filename = Path.home() / 'MEASURES/measurement.xlsx'

def cb_status(msg):
    global goal_status
    status_array = msg.status_list
    try:
        goal_status = status_array[len(status_array) - 1].status
    except:
        # rospy.loginfo("status not received yet")
        goal_status = 0

def cb_cmd_vel(msg):
    global cmd_vel, cmd_vel_received
    cmd_vel = msg
    cmd_vel_received = True


def cb_odom(msg):
    global odom, top_speed
    odom = msg
    top_speed = max(top_speed, msg.twist.twist.linear.x)


movement = 0  # [0: not started, 1: navigating, 2: rotating,\\
3: backwards, 4:stuck]

cmd_vel = Twist()
odom = Odometry()
goal_status = 0
nav_time = 0
nav_dist = 0
nav_speed_integral = 0
nav_avg_speed = 0
top_speed = 0

cmd_vel_received = False
fw_time = 0
fw_speed_integral = 0
fw_avg_speed = 0

rot_time = 0
rot_speed_integral = 0
rot_avg_speed = 0

back_time = 0
stuck_time = 0

stopped = True
rospy.Subscriber(ROSTopics.GOAL_STATUS.name, ROSTopics.GOAL_STATUS.data_class, 
cb_status)
rospy.Subscriber(ROSTopics.CMD_VEL_MUX_OUT.name, ROSTopics.CMD_VEL_MUX_OUT.data_class, 
cb_cmd_vel)
rospy.Subscriber(ROSTopics.T265_ODOM.name, ROSTopics.T265_ODOM.data_class, cb_odom)


while not rospy.is_shutdown():
    last_time = rospy.Time.now()
    last_pos = [odom.pose.pose.position.x, odom.pose.pose.position.y]
    while not goal_status == 0:  # if navigation started at least once
        dt = (rospy.Time.now() - last_time).to_sec()
        ds = sqrt((odom.pose.pose.position.x - last_pos[0]) ** 2 + 
        +(odom.pose.pose.position.y - last_pos[1]) ** 2)
        last_time = rospy.Time.now()
        last_pos = [odom.pose.pose.position.x, odom.pose.pose.position.y]
        # print("cycle - ", last_time.to_sec())

        if goal_status == 1:  # if in navigation
            if stopped:
                stopped = False

            # GENERAL IF MOVE BASE IS NAVIGATING
            nav_time += dt
            nav_dist += ds
            nav_speed_integral += odom.twist.twist.linear.x * dt
            nav_avg_speed = nav_speed_integral / nav_time

            # IF ROBOT IS REQUESTING FORWARD MOVEMENT
            if cmd_vel_received and cmd_vel.linear.x > 0.05:  
            # if received cmd_vel requesting of going forward
                if movement != 1:
                    rospy.loginfo("Navigating...")
                    movement = 1
                fw_time += dt
                fw_speed_integral += odom.twist.twist.linear.x * dt
                fw_avg_speed = fw_speed_integral / fw_time

            # IF ROBOT IS ROTATING ON PLACE
            if cmd_vel_received and abs(cmd_vel.linear.x) < 0.05 and abs(
                    cmd_vel.angular.z) >= 0.1:  
                    # if received cmd_vel requesting of going forward
                if movement != 2:
                    rospy.loginfo("Rotating on place...")
                    movement = 2
                rot_time += dt
                rot_speed_integral += abs(odom.twist.twist.angular.z) * dt
                rot_avg_speed = rot_speed_integral / rot_time

            # IF ROBOT IS MOVING BACKWARDS
            if cmd_vel_received and odom.twist.twist.linear.x < - 0.05:  
            # measuring time while going backwards
                if movement != 3:
                    rospy.loginfo("Going backwards...")
                    movement = 3
                back_time += dt

            # STUCK ON PLACE
            # if cmd_vel_received and abs(odom.twist.twist.linear.x) < 0.05 
            and abs(odom.twist.twist.angular.z) < 0.05:
            if cmd_vel_received and abs(odom.twist.twist.linear.x) < 0.05
            and abs(odom.twist.twist.angular.z) < 0.1 \
                    and cmd_vel.angular.z < 0.1:
                if movement != 4:
                    rospy.loginfo("Stuck...")
                    movement = 4
                stuck_time += dt

            cmd_vel_received = False  # resetting to False to check if cmd_vel is still 
            received in the next loop

        elif goal_status == 3 and not stopped:  # stopped to print results just once
            rospy.loginfo("Robot arrived to goal")
            # PRINT SPEEDS
            print("nav_avg_speed: ", round(nav_avg_speed, 2),
        
                  " | mov_fw_avg_speed:", round(fw_avg_speed, 2),
                  " | rot_avg_speed:", round(rot_avg_speed, 2),
                  " | nav_dist:", round(nav_dist, 2),
                  " | top_speed:", round(top_speed, 2))
            # PRINT TIMES
            print("nav_time:      ", round(nav_time, 2),
                  " | moving_fw_time:  ", round(fw_time, 2),
                  " | rotating_time:", round(rot_time, 2),
                  " | back_time:", round(back_time, 2),
                  " | stuck_time:", round(stuck_time, 2))
            List = [nav_avg_speed, fw_avg_speed, rot_avg_speed, nav_dist,
                    top_speed, nav_time, fw_time, rot_time, back_time, stuck_time]
            List = [round(num, rounding) for num in List]

            try:
                wb = load_workbook(filename)
                ws = wb.worksheets[0]  # select first worksheet
            except FileNotFoundError:
                headers_row = ['Speed req.', 'tot_avg_speed', 'fw_avg_speed',
                               'rot_avg_speed', 'nav_dist', 'top_speed',
                               'nav_time', 'moving_fw_time', 'rotating_time', 
                               'back_time', 'stuck_time']
                wb = Workbook()
                ws = wb.active
                ws.append(headers_row)
            ws.append(List)
            wb.save(filename)

            stopped = True
        rate.sleep()

\end{minted}
\section{Current Measure}
\begin{minted}[fontsize={\fontsize{9}{10}\selectfont}]{python}
    def _cb_motor_current(self, msg: Float32MultiArray):
        t_now = time.time()
        delta_t = t_now - self._last_time_motor
        # get values of current of the wheels, divide by 10 for hardware reason
        self._l_current = msg.data[0] / 10
        self._r_current = msg.data[1] / 10
        # evaluate the absorbed current in [A*h] for each wheel (Ampere * hour)
        self._r_consume = self._r_current * delta_t / 3600
        self._l_consume = self._l_current * delta_t / 3600
        if not self.differential_drive:
            self._rl_current = msg.data[2] / 10
            self._rr_current = msg.data[3] / 10
        else:
            self._rl_current = 0
            self._rr_current = 0
        self._rl_consume = self._rl_current * delta_t / 3600
        self._rr_consume = self._rr_current * delta_t / 3600
        # evaluate the sum of the absorbed current of each wheel
        total_consume = self._r_consume + self._l_consume + self._rl_consume + self._rr_consume
        self._e_abs += total_consume
        self._last_time_motor = t_now
\end{minted}
\section{Postprocesser}
\begin{minted}[fontsize={\fontsize{9}{10}\selectfont}]{cpp}
 // First, include ros library
#include <ros/ros.h>
 // Include then pcl library required
    #include <pcl_conversions/pcl_conversions.h>
    #include <pcl/point_cloud.h>
    #include <pcl/point_types.h>
    #include <pcl/filters/voxel_grid.h>
    #include <pcl/filters/statistical_outlier_removal.h>
    #include <pcl/filters/statistical_outlier_removal.h>
    #include <pcl/io/pcd_io.h>

 // Include PointCloud2 message
 #include <sensor_msgs/PointCloud2.h>

 // Topics (corrected, needs to be checked)
 static const std::string IMAGE_TOPIC = "/d400_lb/depth/color/points";
 static const std::string PUBLISH_TOPIC = "/pointcloudfiltered";

 using namespace std::chrono;

 // ROS Publisher
 ros::Publisher pub;
 int i;

 void cloud_cb(const sensor_msgs::PointCloud2ConstPtr& cloud_msg)
 {

     auto start = high_resolution_clock::now();
     // Container for original & filtered data
     pcl::PCLPointCloud2* cloud = new pcl::PCLPointCloud2;
     pcl::PCLPointCloud2ConstPtr cloudPtr(cloud);
     pcl::PCLPointCloud2 cloud_filtered;

     // Convert to PCL data type
     pcl_conversions::toPCL(*cloud_msg, *cloud);
     //filtering with StatisticalOutlierRemoval
     pcl::StatisticalOutlierRemoval<pcl::PCLPointCloud2> sor;
     sor.setInputCloud(cloudPtr);
     sor.setMeanK(9);
     sor.setStddevMulThresh(1.0);
     sor.filter(cloud_filtered);
     // Convert to ROS data type
     sensor_msgs::PointCloud2 output;
     pcl_conversions::moveFromPCL(cloud_filtered, output);
     // Publish the data
     pub.publish (output);
     auto stop = high_resolution_clock::now();
     auto duration = duration_cast<microseconds>(stop - start);
     std::cout << duration.count() << std::endl;
 }
 int main (int argc, char** argv)
 {
     // Initialize the ROS Node ""
     ros::init (argc, argv, "subscriber_filter_republisher");
     ros::NodeHandle nh;
     // Create a ROS Subscriber to IMAGE_TOPIC with a queue_size of 1 and a callback 
     function to cloud_cb
     ros::Subscriber sub = nh.subscribe(IMAGE_TOPIC, 1, cloud_cb);
     // Create a ROS publisher to PUBLISH_TOPIC with a queue_size of 1
     pub = nh.advertise<sensor_msgs::PointCloud2>(PUBLISH_TOPIC, 1);
     // Spin
     ros::spin();
     // Success
     return 0;
 }
\end{minted}

% LIST OF FIGURES
\listoffigures

% LIST OF TABLES
\listoftables

% LIST OF SYMBOLS
% Write out the List of Symbols in this page
\chapter*{List of Symbols} % You have to include a chapter for your list of symbols (
\begin{table}[H]
    \centering
    \begin{tabular}{lll}
        \textbf{Variable} & \textbf{Description} & \textbf{SI unit} \\\hline\\[-9px]
        $\omega$ & angular speed & rad/s \\[2px]
        $V$ & linear speed & m/s \\[2px]
    \end{tabular}
\end{table}

% ACKNOWLEDGEMENTS
\chapter*{Acknowledgements}
Lo sviluppo di questa tesi si pone come un traguardo per me, alla fine di un percorso
iniziato ormai tre anni fa quando decisi d'intraprendere il passaggio da economia a ingegneria.


Desidero quindi ringraziare prima di tutto la mia famiglia, che da sempre mi supporta,
credendo tanto quanto me in questo progetto. Senza il loro appoggio non sarei arrivato
fino a qui. Una dedica in particolare va a mio nonno, per essere stato per me un esempio
di dedizione al lavoro e forza di determinazione.

Ringrazio le persone che mi hanno accompagnato in questi anni e che tutt’ora
mi sono vicine ogni giorno, con le quali ho condiviso momenti importanti che spero di ricordare a lungo.

Grazie ad Oversonic Robotics, a Fabio Puglia per avermi dato l’opportunità di lavorare
a un progetto così ambizioso, all’ing. Lorenzo Romanini per il supporto e a tutte le
persone incontrate a Besana che hanno alleggerito le giornate di lavoro.


Vorrei esprimere inoltre la mia gratitudine al professor Matteo Matteucci, per la sua
pazienza e la sua disponibilità nell’affiancarmi in questa tesi.

Ringrazio il Politecnico di Milano e il professor Paolo Bolzern per avermi dato la possibilità d'intraprendere questo percorso, anche quando il
mio background non lo consentiva.
Ringrazio infine la città di Milano per avermi accolto e fatto crescere dal punto di vista
umano e professionale in questi anni.



\cleardoublepage

\end{document}
