\\
\\
\\
\begin{center}
    \textit{<<In the twenty-first century the \\
    robot will take the place which \\
    slave labor occupied in ancient \\ 
    civilization. There is no reason at \\ 
    all why most of this should not \\
    come to pass in less than a century, \\
    freeing mankind to pursue its \\
    higher aspirations.>>} \\ 
            \text{Nikola Tesla (1856 - 1943) }
\end{center}


\begin{center}
    \textit{<<Robots of the world! \\
    The power of
man has fallen!\\ A new world has
arisen:\\ the Rule of the Robots!
March!>>}\\
    \text{Rossum's Universal Robot (1920)}
\end{center}

Man has always spent his life working. Dangerous and degrading work has been the cause of death for many people for centuries. 
In this sense, there has always been a tendency to try to relieve man of the heaviest jobs by looking for machines or automatic systems to replace him.
In a sense, with the advent of the industrial revolution, we witnessed the first real process of robotizing in history.
On the other hand, with the evolution of discoveries in the medical field, the desire and curiosity arose in man to try to clone himself, artificially constructing his own like.
It is here that these two needs and tendencies come together in what we now call humanoid robots.
Indeed, humanoid robots are designed and built to replace humans in the most physical and repetitive tasks, in order to ensure greater well-being.

\newpage

\section{Introduction}
In recent years, the general public has become increasingly interested in robots and robotics research. New developments, e.g. robotic competitions, which "[push] beyond the boundaries of current technological
systems" (such as Defense Advanced Research Projects Agency (DARPA) in the
United States), especially in the area of robotics, have promised and delivered
fully integrated systems, \citet{robocomp}




